% !TEX TS-program = pdflatex
% !TEX encoding = UTF-8 Unicode

% This file is a template using the "beamer" package to create slides for a talk or presentation
% - Giving a talk on some subject.
% - The talk is between 15min and 45min long.
% - Style is ornate.

% MODIFIED by Jonathan Kew, 2008-07-06
% The header comments and encoding in this file were modified for inclusion with TeXworks.
% The content is otherwise unchanged from the original distributed with the beamer package.

\documentclass{beamer}


% Copyright 2004 by Till Tantau <tantau@users.sourceforge.net>.
%
% In principle, this file can be redistributed and/or modified under
% the terms of the GNU Public License, version 2.
%
% However, this file is supposed to be a template to be modified
% for your own needs. For this reason, if you use this file as a
% template and not specifically distribute it as part of a another
% package/program, I grant the extra permission to freely copy and
% modify this file as you see fit and even to delete this copyright
% notice. 


\mode<presentation>
{
  \usetheme{Madrid}
  % or ...

 \usecolortheme{dolphin}

  \setbeamercovered{transparent}
  % or whatever (possibly just delete it)
}


\usepackage[english]{babel}
% or whatever

\usepackage[utf8]{inputenc}
% or whatever

\usepackage{times}
\usepackage[T1]{fontenc}
% Or whatever. Note that the encoding and the font should match. If T1
% does not look nice, try deleting the line with the fontenc.


\title[First Order Lower Units] % (optional, use only with long paper titles)
{Towards the Compression of First-Order Resolution Proofs by Lowering Unit Clauses}

%\subtitle{Presentation Subtitle}

\author[Gorzny, Woltzenlogel Paleo] % (optional, use only with lots of authors)
{J. Gorzny\inst{1} \and B. Woltzenlogel Paleo\inst{2}}
% - Use the \inst{?} command only if the authors have different
%   affiliation.

\institute[] % (optional, but mostly needed)
{
  \inst{1}%
  University of Victoria
  \and
  \inst{2}%
  Vienna University of Technology}
% - Use the \inst command only if there are several affiliations.
% - Keep it simple, no one is interested in your street address.

\date[CADE15] % (optional)
{6 August 2015}

%\subject{Talks}
% This is only inserted into the PDF information catalog. Can be left
% out. 



% If you have a file called "university-logo-filename.xxx", where xxx
% is a graphic format that can be processed by latex or pdflatex,
% resp., then you can add a logo as follows:

% \pgfdeclareimage[height=0.5cm]{university-logo}{university-logo-filename}
% \logo{\pgfuseimage{university-logo}}



% Delete this, if you do not want the table of contents to pop up at
% the beginning of each subsection:
%\AtBeginSubsection[]
%{
%  \begin{frame}<beamer>{Outline}
%    \tableofcontents[currentsection,currentsubsection]
%  \end{frame}
%}


% If you wish to uncover everything in a step-wise fashion, uncomment
% the following command: 

%\beamerdefaultoverlayspecification{<+->}


\begin{document}

\begin{frame}
  \titlepage
\end{frame}

%\begin{frame}{Outline}
%  \tableofcontents
  % You might wish to add the option [pausesections]
%\end{frame}


% Since this a solution template for a generic talk, very little can
% be said about how it should be structured. However, the talk length
% of between 15min and 45min and the theme suggest that you stick to
% the following rules:  

% - Exactly two or three sections (other than the summary).
% - At *most* three subsections per section.
% - Talk about 30s to 2min per frame. So there should be between about
%   15 and 30 frames, all told.

\section{Introduction}

%\subsection[Short First Subsection Name]{First Subsection Name}

\begin{frame}{Proof Compression Motivation}
an accessible, good motivational example for proof compression
\end{frame}

\begin{frame}{(Propositional) Proofs}
%introduction to proofs as we're going to see them, ideally formally and with a small example

\begin{definition}[Proof]
A directed acyclic graph $\langle V,E,\Gamma \rangle$, where
\begin{itemize}
\item $V$ is a set of nodes
\item $E$ is a set of edges labeled by literals
\item $\Gamma$ (the proof clause) is inductively constructible using \emph{axiom} and \emph{resolution} nodes
\end{itemize}
\end{definition}

\begin{definition}[Axiom]
A proof with a single node (so $E=\emptyset$)
\end{definition}
\end{frame}

\begin{frame}{(Propositional) Resolution}
%a quick introduction to resolution, with a small propositional example
\begin{definition}[Resolution]
Given two proofs $\psi_L$ and $\psi_R$ with conclusions $\Gamma_L$ and $\Gamma_R$ with some literal $l$ such that $\overline{l}\in \Gamma_L$ and $l\in \Gamma_R$, the resolution proof $\psi$ of $\psi_L$ and $\psi_R$ on $l$, denoted $\psi=\psi_L \psi_R$ is such that:
\begin{itemize}
\item $\psi$'s nodes are the union of the nodes of $\psi_L$ and $\psi_R$, and a new root node
\item there is an edge from $\rho(\psi)$ to $\rho(\psi_L)$ labeled with $\overline{l}$
\item there is an edge from $\rho(\psi)$ to $\rho(\psi_R)$ labeled with $l$
\item $\psi$'s conclusion is $(\Gamma_L\setminus\{\overline{l}\})\cup(\Gamma_R\setminus\{l\})$
\end{itemize}
\end{definition}
\end{frame}

\begin{frame}{A Propositional Proof}
a small example to illustrate the definitions from the last two slides\\
the example should be redundant, so that we can show it again after the next slide in it's more minimal state\\
ideally minimized via LU, so that we can show the transformation later
\end{frame}


\begin{frame}{Deletion}
how deleting subproofs or edges in proofs affect them
\end{frame}

\begin{frame}{Redundancy}
types of redundancy we hope to remove, small examples (before/after proofs; not animated)
\end{frame}

\begin{frame}{First-Order Proofs}
%key differences from propositional case; example of first order proofs
\begin{definition}[First-Order Proof]
A directed acyclic graph $\langle V,E,\Gamma \rangle$, where
\begin{itemize}
\item $V$ is a set of nodes
\item $E$ is a set of edges labeled by literals {\bf and substitutions}
\item $\Gamma$ (the proof clause) is inductively constructible using \emph{axiom}, {\bf\emph{(first order) resolution}}, {\bf and \emph{contraction}} nodes
\end{itemize}
\end{definition}

Axioms are unchanged
\end{frame}

\begin{frame}{Substitutions and Unifiers}
\begin{definition}[Substitution]
A mapping $\{X_1\setminus t_1, X_2\setminus t_2,\ldots\}$ from variables $X_1,X_2,\ldots$ to terms $t_1,t_2,\ldots$
\end{definition}

%example here

\begin{definition}[Unifier]
A set of literals in a substitution that makes all literals in the set equal
\end{definition}

%example here
\end{frame}

\begin{frame}{First Order (Unifying) Resolution}
%definition (incl. mgu); ?
%example of unifying resolution ?
\begin{definition}[First Order Resolution]
Given two proofs $\psi_L$ and $\psi_R$ with conclusions $\Gamma_L$ and $\Gamma_R$ with some literal $l$ such that $l_L\in \Gamma_L$ and $l_R\in \Gamma_R$, and $\sigma_L$ and $\sigma_R$ are substitutions usch that $l_L\sigma_L=\overline{l_R}\sigma_R$, and the variables in $(\Gamma_L\setminus l_L)\sigma_L$ and $(\Gamma_R\setminus l_R)\sigma_R$ are disjoint, then the resolution proof $\psi$ of $\psi_L$ and $\psi_R$ on $l$, denoted $\psi=\psi_L \psi_R$ is such that:
\begin{itemize}
\item $\psi$'s nodes are the union of the nodes of $\psi_L$ and $\psi_R$, and a new root node
\item there is an edge from $\rho(\psi)$ to $\rho(\psi_L)$ labeled with $l_L$ and $\sigma_L$
\item there is an edge from $\rho(\psi)$ to $\rho(\psi_R)$ labeled with $l_R$ and $\sigma_R$
\item $\psi$'s conclusion is $(\Gamma_L\setminus l_L)\sigma_L\cup (\Gamma_R\setminus l_R)\sigma_R$
\end{itemize}
\end{definition}
\end{frame}

\begin{frame}{Contraction}
%definition; small example
\begin{definition}[Contraction]
If $\psi'$ is a proof and $\sigma$ is a unifier of $\{l_1,\ldots,l_n\}\subset \Gamma'$, then a contraction $\psi$ is a proof where
\begin{itemize}
\item $\psi$'s nodes are the union of the nodes of $\psi'$ and a new node $v$
\item There is an edge from $\rho(\psi')$ to $v$ labeled with $\{l_1,\ldots,l_n\}$ and $\sigma$
\item The conclusion is $(\Gamma'\setminus \{l_1,\ldots,l_n\})\sigma \cup \{l\}$, where $l=l_k\sigma$ for $k\in \{1,\ldots,n\}$
\end{itemize}
\end{definition}
\end{frame}

\section{Propositional Algorithm}

\begin{frame}{LowerUnits}
brief high level description; complexity\\
probably not pseudo-code
\end{frame}

\begin{frame}{Propositional Example}
quick, clear example of LU (animated), perhaps showing how one of the redundancies described before is fixed
\end{frame}

\section{First-Order Algorithm}

\begin{frame}{First Order Challenges I}
example 1 demonstrated
\end{frame}

\begin{frame}{First Order Challenges II}
example 2 demonstrated; definition of pre-deletion unification property
\end{frame}

\begin{frame}{First Order Challenges III}
example 2 demonstrated; definition of post-deletion unification property
\end{frame}

\begin{frame}{First Order Lower Units Ideas/Principles}
briefly mention all ideas, e.g. quadratic time naive approach to deal with both properties
\end{frame}

\begin{frame}{Simple/Greedy First Order Lower Units}
introduce simpler idea, make compromises explicit and list benefits\\
high level description\\
(probably not pseudo-code, but list \# of traversals, complexity, etc)
\end{frame}

\begin{frame}{First Order Example}
small, animated example
\end{frame}

\section{Experiment Results}

\begin{frame}{Experiment Setup}
proof sources, systems used, etc.
\end{frame}

\begin{frame}{Results I}
at least one or two of the more informative graphs
\end{frame}

\begin{frame}{Results II}
text summary of results (numbers, percentages, times, etc)
\end{frame}

\section*{Summary}

\begin{frame}{Conclusion}
summary\\
future work (FORPI)\\
source link\\
\end{frame}


\end{document}


