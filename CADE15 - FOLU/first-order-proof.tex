{
  \setbeamercolor{background canvas}{bg=lightgray}
\begin{frame}{Deletion}
%how deleting subproofs or edges in proofs affect them
Deletion of an edge\\
\begin{itemize}
\item The resolvent is replaced by the other premise
\item Some subsequent resolutions may have to be deleted too
\end{itemize}
\vspace{1cm}
Deletion of a subproof $\psi$\\
\begin{itemize}
\item Deletion of every edge coming to $\rho(\psi)$
\item The operation is commutative and associative
\end{itemize}
\end{frame}
}

%{
%  \setbeamercolor{background canvas}{bg=lightgray}
%\begin{frame}{Redundancy}
%types of redundancy we hope to remove, small examples (before/after proofs; not animated)
%\end{frame}
%}

{
  \setbeamercolor{background canvas}{bg=lightgray}
\begin{frame}{First-Order Proofs}
%key differences from propositional case; example of first-order proofs
\begin{definition}[First-Order Proof]
A directed acyclic graph $\langle V,E,\Gamma \rangle$, where
\begin{itemize}
\item $V$ is a set of nodes
\item $E$ is a set of edges labeled by literals {\bf and substitutions}
\item $\Gamma$ (the proof clause) is inductively constructible using \emph{axiom}, {\bf\emph{(first-order) resolution}}, {\bf and \emph{contraction}} nodes
\end{itemize}
\end{definition}
\vspace{0.5cm}
Axioms are unchanged
\end{frame}
}

{
  \setbeamercolor{background canvas}{bg=lightgray}
\begin{frame}{Substitutions and Unifiers}
\begin{definition}[Substitution]
A mapping $\{X_1\setminus t_1, X_2\setminus t_2,\ldots\}$ from variables $X_1,X_2,\ldots$ to terms $t_1,t_2,\ldots$
\end{definition}

%example here

\begin{definition}[Unifier]
A substitution that makes two terms equal when applied to them.
\end{definition}

%example here
\end{frame}
}

{
  \setbeamercolor{background canvas}{bg=lightgray}
\begin{frame}{First-Order (Unifying) Resolution}
%definition (incl. mgu); ?
%example of unifying resolution ?
\begin{definition}[First-Order Resolution]
Given two proofs $\psi_L$ and $\psi_R$ with conclusions $\Gamma_L$ and $\Gamma_R$ with some literal $l$ such that $l_L\in \Gamma_L$ and $l_R\in \Gamma_R$, and $\sigma_L$ and $\sigma_R$ are substitutions usch that $l_L\sigma_L=\overline{l_R}\sigma_R$, and the variables in $(\Gamma_L\setminus l_L)\sigma_L$ and $(\Gamma_R\setminus l_R)\sigma_R$ are disjoint, then the resolution proof $\psi$ of $\psi_L$ and $\psi_R$ on $l$, denoted $\psi=\psi_L \psi_R$ is such that:
\begin{itemize}
\item $\psi$'s nodes are the union of the nodes of $\psi_L$ and $\psi_R$, and a new root node
\item there is an edge from $\rho(\psi)$ to $\rho(\psi_L)$ labeled with $l_L$ and $\sigma_L$
\item there is an edge from $\rho(\psi)$ to $\rho(\psi_R)$ labeled with $l_R$ and $\sigma_R$
\item $\psi$'s conclusion is $(\Gamma_L\setminus l_L)\sigma_L\cup (\Gamma_R\setminus l_R)\sigma_R$
\end{itemize}
\end{definition}
\end{frame}
}

{
  \setbeamercolor{background canvas}{bg=lightgray}
\begin{frame}{Contraction}
%definition; small example
\begin{definition}[Contraction]
If $\psi'$ is a proof and $\sigma$ is a unifier of $\{l_1,\ldots,l_n\}\subset \Gamma'$, then a contraction $\psi$ is a proof where
\begin{itemize}
\item $\psi$'s nodes are the union of the nodes of $\psi'$ and a new node $v$
\item There is an edge from $\rho(\psi')$ to $v$ labeled with $\{l_1,\ldots,l_n\}$ and $\sigma$
\item The conclusion is $(\Gamma'\setminus \{l_1,\ldots,l_n\})\sigma \cup \{l\}$, where $l=l_k\sigma$ for $k\in \{1,\ldots,n\}$
\end{itemize}
\end{definition}
\end{frame}
}