\documentclass[compress]{beamer}
\usepackage{etex}

\setbeamercovered{transparent}
\setbeamertemplate{navigation symbols}{}

\definecolor{vertmoyen}{HTML}{A9CB60}
\definecolor{vertclair}{HTML}{D3F192}
\definecolor{vertfonce}{HTML}{347003}
\definecolor{deletecol}{HTML}{3B8A13}
\definecolor{addcolor} {HTML}{9E0270}

\usecolortheme[named=vertfonce]{structure}

\setbeamercolor{section in head/foot}{fg=vertfonce,bg=vertmoyen}
\setbeamercolor{subsection in head/foot}{bg=vertclair}

\setbeamercolor{author in head/foot}{fg=vertfonce, bg=vertclair}
\setbeamercolor{title in head/foot}{fg=vertfonce, bg=vertmoyen}
\setbeamercolor{date in head/foot}{fg=vertfonce, bg=vertclair}

\setbeamercolor{separation line}{bg=vertfonce}


\setbeamercolor{titlelike}{bg=vertclair}
\setbeamerfont{frametitle}{size=\normalsize}

\makeatletter

\defbeamertemplate*{frametitle}{jogo theme}
{
  \ifbeamercolorempty[bg]{frametitle}{}{\nointerlineskip}%
  \ifx\insertsectionhead\@empty\usebeamerfont{headline}\vspace{-3.45ex}\fi%
  \@tempdima=\textwidth%
  \advance\@tempdima by\beamer@leftmargin%
  \advance\@tempdima by\beamer@rightmargin%
  \begin{beamercolorbox}[left,wd=\the\@tempdima]{frametitle}
    \usebeamerfont{frametitle}%
%    \vbox{}\vskip-0.5ex%
%    \if@tempswa\else\csname beamer@fte#1\endcsname\fi%
    \strut\hspace{1em}\insertframetitle\strut\par%
    {%
      \ifx\insertframesubtitle\@empty%
      \else%
      {\usebeamerfont{framesubtitle}\usebeamercolor[fg]{framesubtitle}\insertframesubtitle\strut\par}%
      \fi
    }%
%    \vskip-1ex%
%    \if@tempswa\else\vskip-.3cm\fi% set inside beamercolorbox... evil here...
  \end{beamercolorbox}%
}

\defbeamertemplate*{headline}{jogo theme}
{%
    \leavevmode%
  \ifx\insertsectionhead\@empty\vspace{3.35ex}%
  \else%
  {%
    \vspace{0.1ex}%
    \hbox{%
    \begin{beamercolorbox}[wd=.5\paperwidth,ht=2.25ex,dp=1ex,right]{section in head/foot}%
      \usebeamerfont{section in head/foot}\insertsectionhead\hspace*{2ex}
    \end{beamercolorbox}%
    \begin{beamercolorbox}[wd=.5\paperwidth,ht=2.25ex,dp=1ex,left]{subsection in head/foot}%
      \usebeamerfont{subsection in head/foot}\hspace*{2ex}\insertsubsectionhead
    \end{beamercolorbox}%
    }%
    \vskip-.1ex%
    \begin{beamercolorbox}[wd=\paperwidth,colsep=.05ex]{separation line}
    \end{beamercolorbox}
  }%
  \fi
}

\defbeamertemplate*{footline}{jogo theme}
{
  \leavevmode%
%    \begin{beamercolorbox}[wd=\paperwidth,colsep=.05ex]{separation line}
%    \end{beamercolorbox}
  \hbox{%
  \begin{beamercolorbox}[wd=.26\paperwidth,ht=2.25ex,dp=1ex,left]{author in head/foot}%
    \usebeamerfont{author in head/foot}\hspace*{1ex}\insertshortauthor
  \end{beamercolorbox}%
  \begin{beamercolorbox}[wd=.48\paperwidth,ht=2.25ex,dp=1ex,center]{title in head/foot}%
    \usebeamerfont{title in head/foot}\insertshorttitle
  \end{beamercolorbox}%
  \begin{beamercolorbox}[wd=.26\paperwidth,ht=2.25ex,dp=1ex,right]{date in head/foot}%
    \usebeamerfont{date in head/foot}\insertshortdate{}\hspace{2em}
    \insertframenumber{} / \inserttotalframenumber\hspace*{1ex} 
  \end{beamercolorbox}}%
  \vskip0pt%
}

\makeatother

\usepackage[utf8]{inputenc}
\usepackage[T1]{fontenc}

\usepackage{kpfonts}

\usepackage{amsmath}

\newcommand{\recalltoc}{\frame{\tableofcontents[sectionstyle=show/shaded,subsectionstyle=show/show/hide]}}

\newcommand{\dual}[1]{\ensuremath{\bar{#1}}}
\newcommand{\dn}[2]{#1 \setminus (#2)}
\newcommand{\set}[2]{\ensuremath{\left\{#1\mid#2\right\}}}

\newenvironment<>{subpart}[1]
{ \begin{block}#2{#1}
  \begin{itemize}
}{
  \end{itemize}
  \end{block}
}

\newcommand<>{\singleline}[1]{\begin{block}{\only#2#1}\end{block}}

\newcommand{\bottomup}{\item[$\uparrow$]}
\newcommand{\topdown} {\item[$\downarrow$]}
\newcommand{\anydir}  {\item[$\updownarrow$]}

\usepackage{bussproofs}
\newcommand{\Resolution}[2]{\RightLabel{\footnotesize{$#1$}} \BinaryInfC{$#2$}}

\usepackage[vlined]{algorithm2e}


\usepackage{tikz}
\usetikzlibrary{positioning}

\tikzstyle{proof edge}=[->,thick,cap=round]
\tikzstyle{deleted edge}=[proof edge, dashed, color=deletecol]

\newcommand{\proofnode}[3][]{
  \node [anchor=mid, #1] (#2) {#3}
}

\newcommand{\rootnode}[1][]{
  \proofnode[#1]{root}{$\bot$}
}

\newcommand<>{\edgewithlabel}[3]{
  \draw#4 [proof edge, color=vertfonce] (#1) -- (#2) node [above, pos=0.3] {\footnotesize #3}
}

\newcommand<>{\drawchildren}[3]{
  \draw#4 [proof edge] (#1) -- (#2);
  \draw#4 [proof edge] (#1) -- (#3)
}

\newcommand{\addchildren}[5]{
  \proofnode[above left  of=#1]{#2}{#3};
  \proofnode[above right of=#1]{#4}{#5}
}

\newcommand{\withchildren}[5]{
  \addchildren{#1}{#2}{#3}{#4}{#5};
  \drawchildren{#1}{#2}{#4}
}

\newcommand<>{\crossnode}[2][]{
  \draw#3 [color=deletecol,thick,cap=round,#1] (#2.mid) ++(10:0.3) -- ++(190:0.6);
}

\newcommand<>{\marknode}[2][.33cm and .29cm]{
  \draw#3 [color=vertclair,line width=1pt] (#2) ellipse (#1);
}

\usepackage{booktabs}



\title[Compression of Propositional Resolution Proofs]{Compression of Propositional Resolution Proofs by Lowering Subproofs}
\author[J. Boudou, B. Woltzenlogel Paleo]{
  Joseph Boudou\inst{1}
  \and 
  Bruno Woltzenlogel Paleo\inst{2}
}
\institute{
  \inst{1}Universit\'e Paul Sabatier, Toulouse
  \and 
  \inst{2}Vienna University of Technology
}
\date[PCC 2013]{PCC Workshop, 2013}


%%%%%%%%%%%%%%%%%%
% Begin Document %
%%%%%%%%%%%%%%%%%%
\begin{document}

\begin{frame}[plain]
\titlepage
\end{frame}

\begin{frame}{Overview}
\tableofcontents
\end{frame}



\section{Introduction}

\subsection{Propositional resolution calculus}

\begin{frame}
%\begin{subpart}{Literals}
%  \item A \emph{literal} is an atomic formula or the negation of an atomic formula.
%  \item The dual of a literal $\ell$, denoted by $\dual{\ell}$, is such that
%        $\dual{p} = \neg p$ and $\dual{\neg p} = p$.
%\end{subpart}
\begin{subpart}{Clauses}
  \item A \emph{clause} is a disjunctive set of literals.
  \item The empty clause corresponds to false.% and is denoted by $\bot$.
  \item Tautologies contain both a literal $\ell$ and its dual $\dual{\ell}$.
\end{subpart}
%\begin{subpart}{Instance}
%\end{subpart}
  \begin{subpart}{SAT context}
%    \item Axioms are instance's clauses.
    \item Proofs of unsatisfiability.
    \item Tautologies are prohibited.
%    \item An \emph{instance} is a conjunctive set of clauses.
  \end{subpart}
  \begin{block}{Robinson's Resolution Principle}
    \vspace{-.5em}
    \begin{prooftree}
      \AxiomC{$\Gamma, \dual{\ell}$}
      \AxiomC{$\ell, \Delta$}
      \Resolution{\ell}{\Gamma, \Delta}
    \end{prooftree}
  \end{block}
\end{frame}

\subsection{Proofs as DAG}

\begin{frame}
  \begin{block}{Proof as a tree}
    \vspace{-.5em}
    \begin{prooftree}
      \AxiomC{$\dual{b}, c$}
      \AxiomC{$\dual{a}, b$}
      \Resolution{b}{\dual{a}, c}
      \AxiomC{$\dual{a}, b$}
      \AxiomC{$\dual{a}, \dual{b}, \dual{c}$}
      \Resolution{\dual{b}}{\dual{a}, \dual{c}}
      \Resolution{\dual{c}}{\dual{a}}
      \AxiomC{$a$}
      \Resolution{a}{\bot}
    \end{prooftree}
  \end{block}
  \begin{block}{Proof as a directed acyclic graph (DAG)}
    \vspace{.5em}
      \centering
      \begin{tikzpicture}[node distance=1.2cm]
        \rootnode;
        \withchildren{root} {r0}{\dual{a}}  {unit}{a};
        \withchildren{r0}   {r1}{\dual{a},c} {r2}{\dual{a},\dual{c}};
        \withchildren{r1}   {a0}{\dual{b},c} {low}{\dual{a},b};
        \proofnode[above right of=r2] {a1} {\dual{a},\dual{b},\dual{c}};
        \drawchildren {r2} {low} {a1};
      \end{tikzpicture}
  \end{block}
\end{frame}

\begin{frame}{Proof}
  \begin{definition}[Proof]
    A proof $\psi$ is a directed acyclic graph
    \begin{itemize}
      \item having a root noted $\rho(\psi)$;
      \item with nodes labeled with clauses;
      \item with edges oriented from the resolvent to the premise;
      \item with edges labeled with the premise's literal removed in the resolvent;
      \item which is either an axiom or a resolution proof.
    \end{itemize}
  \end{definition}
  \begin{definition}[Axiom]
    An axiom is a proof with only one node.
  \end{definition}
\end{frame}
  
\subsection{Proofs' operations}

\begin{frame}{Resolution}
  Given two proofs $\varphi_L$ and $\varphi_R$ with conclusion $\Gamma_L$ and $\Gamma_R$ and a
  literal $\ell$ s.t. $\dual{\ell} \in \Gamma_L$ and $\ell \in \Gamma_R$, the resolution proof
  $\psi$ of $\varphi_L$ and $\varphi_R$ on $\ell$, noted $\psi = \varphi_L \odot_\ell \varphi_R$, is
  such that:
  \begin{itemize}
    \item $\psi$'s nodes are the union of $\varphi_L$ and $\varphi_R$ nodes plus a new root node;
    \item there is an edge from $\rho(\psi)$ to $\rho(\varphi_L)$ labeled with $\dual{\ell}$;
    \item there is an edge from $\rho(\psi)$ to $\rho(\varphi_R)$ labeled with $\ell$;
    \item $\psi$'s conclusion is $\left( \Gamma_L \setminus \{\dual{\ell}\} \right) \cup
                                  \left( \Gamma_R \setminus \{\ell\} \right)$.
  \end{itemize}
\end{frame}

%\begin{frame}{Subproofs}
%  \begin{definition}[Subproof]
%    A proof $\psi' = \langle V', E', \lambda_V', \lambda_E', \rho' \rangle$ is a subproof of a proof
%    $\psi = \langle V, E, \lambda_V, \lambda_E, \rho \rangle$ iff there exists an injective
%    function $f$ from $V'$ to $V$ such that: \begin{itemize}
%    \item $\forall \eta' \in V',~ \lambda_V'(\eta') = \lambda_V(f(\eta'))$;
%    \item $\forall (\eta_F', \eta_T') \in E',~ (f(\eta_F'), f(\eta_T')) \in E$ and
%          $\lambda_E'(\eta_F', \eta_T') = \lambda_E(f(\eta_F'), f(\eta_T'))$;
%    \end{itemize}
%  \end{definition}
%  \begin{subpart}{Properties}
%    \item Subproof is a partial order relation on proofs, noted $\sqsubseteq$.
%    \item The relation $\equiv$ such that
%          $\psi' \equiv \psi \Leftrightarrow \psi' \sqsubseteq \psi \wedge \psi \sqsubseteq \psi'$
%          is an equivalence relation on proofs.
%  \end{subpart}
%\end{frame}

\begin{frame}{Deletion}
  \begin{subpart}{Deletion of an edge}
    \item The resolvent is replaced by the other premise.
    \item Some subsequent resolutions may have to be deleted too.
  \end{subpart}
  \begin{subpart}{Deletion of a subproof $\varphi$}
    \item Deletion of every edge coming to $\rho(\varphi)$.
    \item The operation is commutative and associative.
  \end{subpart}
  \begin{block}{Notation}
    $\dn{\psi}{\varphi_1, \ldots, \varphi_n}$ denotes the deletions of subproofs $\varphi_1, \ldots,
    \varphi_n$ from the proof $\psi$.
  \end{block}
\end{frame}



\section{Redundancies and corresponding algorithms}

\recalltoc

\subsection{Vertical redundancy}

\begin{frame}{Regular proof}
  \begin{definition}[Tseitin 1970]
    A proof is regular iff on every path from its root to any of its axiom, any literal
    appears at most once as edge label.
  \end{definition}
  \begin{theorem}[Goerdt 1990]
    Given a set of axioms and a clause $\Gamma$, the smallest regular proof of $\Gamma$ might be exponentially bigger
    than the smallest irregular proof of $\Gamma$.
  \end{theorem}
\end{frame}

\begin{frame}{RecyclePivotsWithIntersection (RPI)}
  \begin{subpart}{Partial Regularization}
    \item Delete an outgoing edge labeled with $\ell$ iff $\dual{\ell}$ appears on \textbf{every}
          path from the root to the node.
  \end{subpart}
  \begin{definition}[Safe literal]
    A literal is safe for a node $\eta$ if it can be added to $\eta$'s clause without changing proof's conclusion.
  \end{definition}
  \begin{subpart}{Two traversals}
    \bottomup Collect safe literals and mark edges to be deleted.
    \topdown Delete edges.
  \end{subpart}
\end{frame}

\begin{frame}
  \begin{columns}
  \column{0.5\textwidth}
    \begin{center}
    \begin{tikzpicture}[node distance=1.5cm]

      \rootnode;
      \addchildren{root} {a4}{$\dual{c}$} {r2}{$c$};
      \draw [proof edge] (root) -- (a4);
      \draw<1> [proof edge] (root) -- (r2);
      \edgewithlabel<2->{root}{r2}{$c$};

      \withchildren{r2} {r1}{$\dual{b},c$} {r0}{$b$};

      \withchildren{r0} {iu}{$a$} {a0}{$\dual{a},b$};

      \addchildren{iu} {a2}{$a,\dual{c}$} {a1}{$a,c$};
      \draw [proof edge] (iu) -- (a2);
      \draw<1> [proof edge] (iu) -- (a1);
      \edgewithlabel<2->{iu}{a1}{$c$};

      \proofnode[above left of=r1]{a3}{$\dual{a},\dual{b},c$};
      \draw [proof edge] (r1) -- (a3);
      \draw [proof edge] (r1) -- (iu);
    \end{tikzpicture}
    \visible<3->{Original proof} \\
    \visible<4>{\small 4 resolutions}
    \end{center}

    \column{0.5\textwidth}
    \only<3->{
    \begin{center}
    \begin{tikzpicture}[node distance=1.5cm]

      \rootnode;
      \withchildren{root} {a4}{$\dual{c}$} {r2}{$c$};

      \withchildren{r2} {r1}{$\dual{b},c$} {r0}{\alt<4>{$b,c$}{$b$}};

      \addchildren{r0} {iu}{$a$} {a0}{$\dual{a},b$};
      \draw [proof edge] (r0) -- (a0);
      \draw<3>[proof edge] (r0) -- (iu);
      \draw<4->[proof edge] (r0) -- (a1);

      \addchildren{iu} {a2}{$a,\dual{c}$} {a1}{$a,c$};
      \draw<3> [deleted edge] (iu) -- (a2);
      \draw<3> [proof edge] (iu) -- (a1);

      \proofnode[above left of=r1]{a3}{$\dual{a},\dual{b},c$};
      \draw [proof edge] (r1) -- (a3);
      \draw<3>[proof edge] (r1) -- (iu);
      \draw<4->[proof edge] (r1) .. controls (iu.south east) .. (a1);

      \crossnode<4>{a2}
      \crossnode<4>{iu}
    \end{tikzpicture}
    \uncover<4>{Compressed proof} \\
    \visible<4>{\small 3 resolutions}
    \end{center}
    }
  \end{columns}
\end{frame}


\subsection{Horizontal redundancy}

\begin{frame}
  \begin{definition}
    A node is an horizontal redundancy iff it has at least two incoming edges labeled with the same
    literal.
  \end{definition}
  \begin{subpart}{Reducing horizontal redundancy}
    \item postponing resolution until resolvents are resolved.
  \end{subpart}
  \begin{example} \vspace{-.7em} \begin{columns}
    \column{0.5\textwidth}
    \begin{center} \begin{tikzpicture}[node distance=1.3cm]
      \rootnode;
      \withchildren{root} {a0}{$\dual{c}$} {r0}{$c$};

      \withchildren{r0} {r2}{$\dual{b},c$} {r1}{$b,c$};

      \addchildren{r2} {a2}{$\dual{a},\dual{b},c$} {ln}{$a,c$};
      \draw [proof edge] (r2) -- (a2);
      \edgewithlabel{r2}{ln}{$a$};

      \proofnode[above right of=r1]{a1}{$\dual{a},b$};
      \edgewithlabel{r1}{ln}{$a$};
      \draw [proof edge] (r1) -- (a1);

      \marknode{ln};
    \end{tikzpicture} \end{center}
    \column{0.5\textwidth}
    \begin{center} \begin{tikzpicture}[node distance=1.3cm]
      \rootnode;
      \withchildren{root} {a0}{$\dual{c}$} {r0}{$c$};

      \withchildren{r0} {r1}{$\dual{a},c$} {ln}{$a,c$};

      \withchildren{r1} {a2}{$\dual{a},\dual{b},c$} {a1}{$\dual{a},b$};

      \marknode{ln};
    \end{tikzpicture} \end{center}
  \end{columns} \end{example}
\end{frame}


\begin{frame}{LowerUnits (LU)}
  \begin{definition}[Unit]
    A unit is a subproof with a conclusion clause having exactly one literal.
  \end{definition}
  \begin{theorem}
    A unit can always be lowered.
  \end{theorem}
  \begin{subpart}{Two traversals}
    \anydir Collect units with more than one resolvent.
    \topdown Delete units and reintroduce them at the bottom of the proof.
  \end{subpart}
\end{frame}

\begin{frame}
  \begin{columns}
  \column{0.5\textwidth}
    \begin{center}
    \begin{tikzpicture}[node distance=1.33cm]

      \rootnode;
      \withchildren{root} {r0}{$\dual{c}$} {r1}{$c$};

      \withchildren{r0} {r2}{$\dual{b},\dual{c}$} {u0}{$b$};

      \proofnode[above right of=r1]{a0}{$\dual{b},c$};
      \drawchildren{r1} {u0}{a0};

      \withchildren{r2} {a1}{$a,\dual{b},\dual{c}$} {u1}{$\dual{a}$};

      \proofnode[above right of=u0]{a2}{$a,b$};
      \drawchildren{u0} {u1}{a2};

    \end{tikzpicture}
    \visible<2->{Original proof} \\
    \visible<8>{\small 5 resolutions}
    \end{center}

    \column{0.5\textwidth}
    \only<2->{
    \begin{center}
    \begin{tikzpicture}[node distance=1.33cm]

      \proofnode {root} {\alt<-2,8>{$\bot$}{\temporal<5>{$\dual{b}$}{$a,\dual{b}$}{}}};
      \addchildren{root} {r0}{\temporal<6>{$\dual{c}$}{}{$a$}} {r1}{\alt<-5>{$c$}{$\dual{a}$}};
      \drawchildren<2,8>{root} {r0}{r1};
      \draw<3-4>[proof edge] (root) .. controls (r0.south west) .. (r2);
      \draw<5>[proof edge] (root) .. controls (r0.south west) and (r2.south west) .. (a1);
      \draw<3-5>[proof edge] (root) .. controls (r1.south east) .. (a0);

      \addchildren{r0} {r2}{\alt<-5>{$\dual{b},\dual{c}$}{$a,\dual{b}$}} {u0}{\alt<-5>{$b$}{$a,b$}};
      \draw<2>[proof edge] (r0) -- (r2);
      \draw<2>[deleted edge] (r0) -- (u0);
      \drawchildren<7->{r0} {r2}{u0};

      \proofnode[above right of=r1]{a0}{\only<-5>{$\dual{b},c$}};
      \draw<2>[deleted edge] (r1) -- (u0);
      \draw<2>[proof edge] (r1) -- (a0);

      \addchildren{r2} {a1}{$a,\dual{b},\dual{c}$} {u1}{\alt<-5>{$\dual{a}$}{$\dual{b},c$}};
      \draw<-4,6->[proof edge] (r2) -- (a1);
      \draw<-3,6->[proof edge] (r2) -- (u1);
      \draw<4>[deleted edge] (r2) -- (u1);

      \proofnode[above right of=u0]{a2}{\only<-5>{$a,b$}};
      \draw<-3>[proof edge] (u0) -- (u1);
      \draw<4>[deleted edge] (u0) -- (u1);
      \draw<-4>[proof edge] (u0) -- (a2);

      \crossnode<3-5>{r0};
      \crossnode<3-5>{r1};
      \crossnode<5>{r2};
      \crossnode<5>{u0};
  
      \marknode<2-4,6>{u0};
      \marknode<5>{a2};

      \marknode<4-5>{u1};
      \marknode<6-7>{r1}

    \end{tikzpicture}
    \uncover<8>{Compressed proof} \\
    \visible<8>{\small 3 resolutions}
    \end{center}
    }
  \end{columns}
\end{frame}



\section{LowerUnivalents}

\recalltoc

\subsection{Principles}

\begin{frame}
  \begin{subpart}{Goals}
    \item Lower more subproofs.
    \item Allow fast combination after RPI.
  \end{subpart}
  \begin{subpart}{Idea}
    \item If a unit with conclusion clause $\{a\}$ is already marked for lowering, a subproof with
      conclusion clause $\{\dual{a}, b\}$ may be lowered too.
  \end{subpart}
\end{frame}

\begin{frame}
%  \begin{definition}[Active literals]
%    In a proof $\psi$, the set of active literals of a subproof $\varphi$ are the labels of edges
%    coming into $\varphi$'s root.
%  \end{definition}
  \begin{definition}[Valent literal]
    In a proof $\psi$, a literal $\ell$ is \emph{valent} for the subproof $\varphi$ iff $\dual{\ell}$
    belongs to the conclusion of $\dn{\psi}{\varphi}$ but not to the conclusion of $\psi$.
  \end{definition}
  \begin{definition}[Univalent subproof]
    A subproof $\varphi$ with conclusion $\Gamma$ is \emph{univalent} w.r.t. a set $\Delta$ of
    literals iff $\varphi$ has exactly one valent literal $\ell$, $\ell \notin \Delta$ and $\Gamma
    \subseteq \Delta \cup \left\{ \ell \right\}$. $\ell$ is called the \emph{univalent literal} of
    $\varphi$ w.r.t.  $\Delta$.
  \end{definition}
  \begin{theorem}
    Given a proof $\psi$, if there is a sequence $U = (\varphi_1 \ldots \varphi_n)$ of $\psi$'s
    subproofs and a sequence $(\ell_1 \ldots \ell_n)$ of literals such that
    $\forall i \in [1 \ldots n]$, $\ell_i$ is the univalent literal of $\varphi_i$ w.r.t.
    $\Delta_{i-1} = \{\dual{\ell_1} \ldots \dual{\ell_{i-1}}\}$, then the conclusion of
    $$ \psi' = \dn{\psi}{U} \odot_{\ell_n} \varphi_n \ldots \odot_{\ell_1} \varphi_1 $$
    subsumes the conclusion of $\psi$.
  \end{theorem}
\end{frame}

\subsection{Algorithm and implementation}

\begin{frame}
  \SetKwData{Univ}{Univalents}
  \SetKwFunction{delete}{delete}
  \SetKw{Let}{let}
  \begin{algorithm}[H]
    \KwIn {a proof $\psi$}
    \KwOut{a compressed proof $\psi'$}
    \BlankLine

    \SetKw{Push}{push}
    \SetKw{Pop} {pop}

    \Univ $\leftarrow \varnothing$ \;
    $\Delta \leftarrow \varnothing$ \;
    \BlankLine

    \For{every subproof $\varphi$, in a top-down traversal \label{line:LUniv:step1begin} }{
      $\psi' \leftarrow \varphi \setminus$\Univ \label{line:LUniv:delete} \;
      \If{$\psi'$ is univalent w.r.t. $\Delta$ \label{line:LUniv:lunivtest} }{
        \Let{$\ell$} be the univalent literal \;
        \Push $\dual{\ell}$ onto $\Delta$ \label{line:LUniv:pushDelta} \;
        \Push $\psi'$     onto \Univ \label{line:LUniv:step1end} \;
      }
    }
    \BlankLine

    \tcp{At this point, $\psi' = \psi \setminus \Univ$}
    \While{\Univ $\neq \varnothing$}{ \label{line:LUniv:reintroducebegin}
      $\varphi \leftarrow$ \Pop from \Univ \;
      $\ell \leftarrow$ \Pop from $\Delta$ \;
      \lIf{$\ell$ in the conclusion of $\psi'$ \label{line:LUniv:testreintroduce} }{
      $\psi' \leftarrow \varphi \odot_\ell \psi'$ \;}
    }
  \end{algorithm}
\end{frame}

\begin{frame}
  $$ \Delta = \temporal<2>{\varnothing}{\{\dual{a}\}}{\{\dual{a},\dual{b}\}} $$
  \vspace{-1em}
  \begin{columns}
  \column{0.5\textwidth}
    \begin{center}
    \begin{tikzpicture}[node distance=1.5cm]

      \rootnode;
      \withchildren{root} {r0}{$\dual{a}$} {u0}{$a$};

      \withchildren{r0} {r1}{$\dual{a},\dual{c}$} {r2}{$\dual{a},c$};

      \withchildren{r1} {a0}{$\dual{a},\dual{b},\dual{c}$} {u1}{$\dual{a},b$};

      \proofnode[above right of=r2]{a1}{$\dual{b},c$};
      \drawchildren{r2} {u1}{a1};

    \end{tikzpicture}
    Original proof \\
    \visible<10>{\small 4 resolutions}
    \end{center}

    \column{0.5\textwidth}
    \begin{center}
    \begin{tikzpicture}[node distance=1.5cm]

      \proofnode{root} {\only<-7,10>{$\bot$}};
      \addchildren{root} {r0}{\alt<-5,9->{$\dual{a}$}{\only<6-7>{$\dual{a},\dual{b}$}}}
                         {u0}{$a$};
      \draw<-6,10>[proof edge] (root) -- (r0);
      \draw<1,10> [proof edge] (root) -- (u0);
      \draw<2-6>[deleted edge] (root) -- (u0);

      \addchildren{r0} {r1}{\alt<-7>{$\dual{a},\dual{c}$}{$\dual{a},\dual{b}$}}
                       {r2}{\alt<-7>{$\dual{a},c$}{$\dual{a},b$}};
      \drawchildren<-3,9->{r0} {r1}{r2};
      \draw<4-7>[proof edge] (r0) .. controls (r1.south west) .. (a0);
      \draw<4>  [proof edge] (r0) -- (r2);
      \draw<5-7>[proof edge] (r0) .. controls (r2.south east) .. (a1);

      \addchildren{r1} {a0}{$\dual{a},\dual{b},\dual{c}$} {u1}{\alt<-7>{$\dual{a},b$}{$\dual{b},c$}};
      \draw<-3,8->[proof edge] (r1) -- (a0);
      \draw<-2,8->[proof edge] (r1) -- (u1);
      \draw<3>  [deleted edge] (r1) -- (u1);

      \proofnode[above right of=r2]{a1}{\only<-7>{$\dual{b},c$}};
      \draw<-2>   [proof edge] (r2) -- (u1);
      \draw<3,4>[deleted edge] (r2) -- (u1);
      \draw<-4>   [proof edge] (r2) -- (a2);

      \marknode<2-> [.31cm and .27cm]{u0};
      \marknode<3-7>[.35cm and .3cm]{u1};
      \marknode<8-> [.35cm and .3cm]{r2};

      \crossnode<4-7>{r1};
      \crossnode<5-7>{r2};
      \crossnode<7>{root};
    \end{tikzpicture}
    \uncover<10>{Compressed proof} \\
    \visible<10>{\small 3 resolutions}
    \end{center}
  \end{columns}
\end{frame}

\subsection{Experiments}

\begin{frame}
  \begin{subpart}{Configuration}
    \item Algorithms implemented in Scala for the Skeptik library.
    \item 5\,000 SMT proofs produced by the VeriT solver.
    \item Experiments performed on the Vienna Scientific Cluster.
  \end{subpart}
  \begin{block}{Results}
    \vspace{1em}
    \centering
    \begin{tabular}{lcc}
      \toprule
      Algorithm              & Compression & Speed   \\
      \midrule
      LowerUnits               &  7.5 \% & 22.4 n/ms \\
      LowerUnivalents          &  8.0 \% & 20.4 n/ms \\
      LU composed after RPI    & 21.7 \% & 15.1 n/ms \\
      LUniv combined after RPI & 22.0 \% & 17.8 n/ms \\    
      \bottomrule
    \end{tabular}
  \end{block}
\end{frame}



\section{Conclusion}

\begin{frame}
  \begin{subpart}{Goals achieved}
    \item LowerUnivalents compresses more than LowerUnits.
    \item LowerUnivalents combines efficiently after RPI.
  \end{subpart}
  \begin{subpart}{Future works}
    \item Combine LowerUnivalents after other algorithms.
    \item Get rid of order dependency.
    \item Lower subproofs just until resolvents are all resolved.
    \item Explore other kind of redundancies.
  \end{subpart}
\end{frame}

\begin{frame}
  \vskip0pt plus 1filll
  \begin{minipage}{\textwidth}
    \singleline{\centering Thank you for your attention.}
    \singleline{\centering Any question ?}
  \end{minipage}
  \vskip0pt plus 1filll
  \begin{minipage}{\textwidth}
    \begin{subpart}{Skeptik}
      \item http://github.com/Paradoxika/Skeptik
    \end{subpart}
  \end{minipage}
\end{frame}

\end{document}
% vim: tw=100
