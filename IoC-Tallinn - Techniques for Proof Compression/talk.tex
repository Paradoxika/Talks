\documentclass[9pt]{beamer}
\mode<presentation>
{
  \usetheme{Warsaw}
  \usecolortheme{crane}
  \setbeamercovered{transparent}
}

%\usepackage{pseudocode}

\usepackage[english]{babel}
\usepackage[latin1]{inputenc}
\usepackage{times}
\usepackage[T1]{fontenc}
\usepackage{verbatim}

\usepackage{amsmath}
\usepackage{amssymb}
\usepackage{amsfonts}
\usepackage{pxfonts}
%\usepackage{dsfont}
%\usepackage{yfonts}
\usepackage{amsthm}
\usepackage{graphicx}

\usepackage{bussproofs}
\usepackage{proof}

\usepackage{fancybox}
\usepackage{fancyvrb}


\usepackage{commands}


% Sequent Calculus Proof Settings
\EnableBpAbbreviations
\def\fCenter{\mbox{\ $\vdash$\ }}


% *************** Some colour definitions ***************
\usepackage{color}

\definecolor{greenyellow}   {cmyk}{0.15, 0   , 0.69, 0   }
\definecolor{yellow}        {cmyk}{0   , 0   , 1   , 0   }
\definecolor{goldenrod}     {cmyk}{0   , 0.10, 0.84, 0   }
\definecolor{dandelion}     {cmyk}{0   , 0.29, 0.84, 0   }
\definecolor{apricot}       {cmyk}{0   , 0.32, 0.52, 0   }
\definecolor{peach}         {cmyk}{0   , 0.50, 0.70, 0   }
\definecolor{melon}         {cmyk}{0   , 0.46, 0.50, 0   }
\definecolor{yelloworange}  {cmyk}{0   , 0.42, 1   , 0   }
\definecolor{orange}        {cmyk}{0   , 0.61, 0.87, 0   }
\definecolor{burntorange}   {cmyk}{0   , 0.51, 1   , 0   }
\definecolor{bittersweet}   {cmyk}{0   , 0.75, 1   , 0.24}
\definecolor{redorange}     {cmyk}{0   , 0.77, 0.87, 0   }
\definecolor{mahogany}      {cmyk}{0   , 0.85, 0.87, 0.35}
\definecolor{maroon}        {cmyk}{0   , 0.87, 0.68, 0.32}
\definecolor{brickred}      {cmyk}{0   , 0.89, 0.94, 0.28}
\definecolor{red}           {cmyk}{0   , 1   , 1   , 0   }
\definecolor{orangered}     {cmyk}{0   , 1   , 0.50, 0   }
\definecolor{rubinered}     {cmyk}{0   , 1   , 0.13, 0   }
\definecolor{wildstrawberry}{cmyk}{0   , 0.96, 0.39, 0   }
\definecolor{salmon}        {cmyk}{0   , 0.53, 0.38, 0   }
\definecolor{carnationpink} {cmyk}{0   , 0.63, 0   , 0   }
\definecolor{magenta}       {cmyk}{0   , 1   , 0   , 0   }
\definecolor{violetred}     {cmyk}{0   , 0.81, 0   , 0   }
\definecolor{rhodamine}     {cmyk}{0   , 0.82, 0   , 0   }
\definecolor{mulberry}      {cmyk}{0.34, 0.90, 0   , 0.02}
\definecolor{redviolet}     {cmyk}{0.07, 0.90, 0   , 0.34}
\definecolor{fuchsia}       {cmyk}{0.47, 0.91, 0   , 0.08}
\definecolor{lavender}      {cmyk}{0   , 0.48, 0   , 0   }
\definecolor{thistle}       {cmyk}{0.12, 0.59, 0   , 0   }
\definecolor{orchid}        {cmyk}{0.32, 0.64, 0   , 0   }
\definecolor{darkorchid}    {cmyk}{0.40, 0.80, 0.20, 0   }
\definecolor{purple}        {cmyk}{0.45, 0.86, 0   , 0   }
\definecolor{plum}          {cmyk}{0.50, 1   , 0   , 0   }
\definecolor{violet}        {cmyk}{0.79, 0.88, 0   , 0   }
\definecolor{royalpurple}   {cmyk}{0.75, 0.90, 0   , 0   }
\definecolor{blueviolet}    {cmyk}{0.86, 0.91, 0   , 0.04}
\definecolor{periwinkle}    {cmyk}{0.57, 0.55, 0   , 0   }
\definecolor{cadetblue}     {cmyk}{0.62, 0.57, 0.23, 0   }
\definecolor{cornflowerblue}{cmyk}{0.65, 0.13, 0   , 0   }
\definecolor{midnightblue}  {cmyk}{0.98, 0.13, 0   , 0.43}
\definecolor{navyblue}      {cmyk}{0.94, 0.54, 0   , 0   }
\definecolor{royalblue}     {cmyk}{1   , 0.50, 0   , 0   }
\definecolor{blue}          {cmyk}{1   , 1   , 0   , 0   }
\definecolor{cerulean}      {cmyk}{0.94, 0.11, 0   , 0   }
\definecolor{cyan}          {cmyk}{1   , 0   , 0   , 0   }
\definecolor{processblue}   {cmyk}{0.96, 0   , 0   , 0   }
\definecolor{skyblue}       {cmyk}{0.62, 0   , 0.12, 0   }
\definecolor{turquoise}     {cmyk}{0.85, 0   , 0.20, 0   }
\definecolor{tealblue}      {cmyk}{0.86, 0   , 0.34, 0.02}
\definecolor{aquamarine}    {cmyk}{0.82, 0   , 0.30, 0   }
\definecolor{bluegreen}     {cmyk}{0.85, 0   , 0.33, 0   }
\definecolor{emerald}       {cmyk}{1   , 0   , 0.50, 0   }
\definecolor{junglegreen}   {cmyk}{0.99, 0   , 0.52, 0   }
\definecolor{seagreen}      {cmyk}{0.69, 0   , 0.50, 0   }
\definecolor{green}         {cmyk}{1   , 0   , 1   , 0   }
\definecolor{forestgreen}   {cmyk}{0.91, 0   , 0.88, 0.12}
\definecolor{pinegreen}     {cmyk}{0.92, 0   , 0.59, 0.25}
\definecolor{limegreen}     {cmyk}{0.50, 0   , 1   , 0   }
\definecolor{yellowgreen}   {cmyk}{0.44, 0   , 0.74, 0   }
\definecolor{springgreen}   {cmyk}{0.26, 0   , 0.76, 0   }
\definecolor{olivegreen}    {cmyk}{0.64, 0   , 0.95, 0.40}
\definecolor{rawsienna}     {cmyk}{0   , 0.72, 1   , 0.45}
\definecolor{sepia}         {cmyk}{0   , 0.83, 1   , 0.70}
\definecolor{brown}         {cmyk}{0   , 0.81, 1   , 0.60}
\definecolor{tan}           {cmyk}{0.14, 0.42, 0.56, 0   }
\definecolor{gray}          {cmyk}{0   , 0   , 0   , 0.50}
\definecolor{black}         {cmyk}{0   , 0   , 0   , 1   }
\definecolor{white}         {cmyk}{0   , 0   , 0   , 0   } 


\newcommand{\hl}[1]{{\color{brickred} #1}}

\newcommand{\mypause}{\pause}
%\newcommand{\mypause}{}

\title[Techniques for Proof Compression]{Techniques for Proof Compression}

\author{Bruno Woltzenlogel Paleo}

\institute[] {
  \inst{}%
}

\date[27.06.13]
{2013 June 27th}

\begin{document}

\begin{frame}
  \titlepage
\end{frame}


\newcommand{\C}{\mathcal{C}}
\newcommand{\CurryHoward}{\mathcal{I}}
\newcommand{\ND}{$\textbf{ND}$}
\newcommand{\NDd}{$\textbf{NDc}$}
%\newcommand{\NDd}{$\textbf{ND}_\textbf{d}$}
\newenvironment{calculus}{\begin{center}\begin{Sbox}\begin{minipage}{0.95\textwidth}}{\end{minipage}\end{Sbox}\fbox{\TheSbox}\end{center}}


\newcommand{\tP}[1]{\xi[#1]}
\newcommand{\tN}[1]{\zeta[#1]}
\newcommand{\eS}{\epsilon}

\newcommand{\AtPosition}{\textrm{At}}

\begin{frame}{Motivations for Proof Compression}
\begin{itemize}
\item Sat/SMT-solvers, ATPs, proof assistants\ldots
	\begin{itemize}
	\item best techniques to find proofs do not necessarily find the best proofs
	\item proofs can be redundant 
	\end{itemize}
\mypause
\item Proof compression techniques may lead to: 
	\begin{itemize}
	\item smaller proof libraries
	\item faster proof checking
	\item smaller unsat cores
	\item better interpolants
	\item easier exchange of knowledge
	\item discovery of interesting mathematical definitions and lemmas
	\end{itemize}
\end{itemize}
\end{frame}

\begin{frame}{Techniques for Proof Compression}
\begin{itemize}
\item Sequent Calculus
	\begin{itemize}
	\item \emph{Cut-elimination}
	\item \emph{Cut-introduction}
	\end{itemize}
\item Natural Deduction
	\begin{itemize}
	\item \alert{\emph{Allowing \emph{contextual} inferences}}
	\end{itemize}
\item Propositional Resolution
	\begin{itemize}
	\item \alert{\emph{Recycle Pivots (with Intersection)}}
	\item \alert{\emph{Lower Units}}
	\item \alert{Reduce\&Reconstruct}
	\item \alert{Split}
	\end{itemize}
\end{itemize}
\end{frame}




% \begin{frame}{Why Deep Inference? Why Natural Deduction?}
% \begin{itemize}
% \item Sat/SMT-solvers and first-order ATPs do reason deeply \\
% \mypause
% \item e.g. preprocessing techniques change formulas deeply:
% $$
% \neg \neg \ex x. P(x) \vee \all y. Q(y)
% \quad \longmapsto \quad
% P(c_{sk}) \vee Q(v)
% $$
% %
% \mypause
% \item ATPs are not trusted and are hard to verify
% \mypause
% \item Verify the output proofs instead
% \mypause
% \item Replay the proofs in trusted proof assistants 
% \mypause
% \item ATP proofs are coarse-grained \\
% 		(several (deep) steps hidden in a single inference)
% \mypause
% \item Finer-grained proofs may be desirable \\ (faster to check/replay, more informative)
% \mypause
% \item Fine-grained ATP proofs would need \alert{deep} inference rules
% \mypause
% \item Popular proof assistants follow a \emph{shallow} \alert{natural deduction} style
% \mypause
% \item Replaying deep inference in shallow natural deduction is \\ unnatural, inefficient and lengthy
% \mypause
% \item Therefore, \alert{contextual natural deduction}!
% \end{itemize}
% \end{frame}





\begin{frame}{Natural Deduction}
\begin{figure}[h!]
\begin{calculus}
\begin{prooftree}
\AXC{$ $} \RightLabel{$axiom$}
\UIC{$ \Gamma, A \seq A$}
\end{prooftree}
\begin{prooftree}
\AXC{$ \Gamma, A \seq B$} \RightLabel{$\imp_I$}
\UIC{$ \Gamma \seq A \imp B$}
\end{prooftree}
\begin{prooftree}
\AXC{$ \Gamma \seq A \imp B $}
		\AXC{$ \Gamma \seq A$} \RightLabel{$\imp_E$}
	\BIC{$ \Gamma \seq B$}
\end{prooftree}
\end{calculus}
\caption{The natural deduction calculus \ND}
\label{figure:ND}
\end{figure}
\end{frame}
 

\begin{frame}{Natural Deduction}{An example: double negation elimination}

Double negation elimination axiom schema:
$$
dne : \neg \neg F \imp F
$$

Deriving $(A \imp B) \imp C$ from $(\neg \neg A \imp B) \imp C$   in {\ND}:

\begin{footnotesize}
\begin{prooftree}
\AXC{$(\neg \neg A \imp B) \imp C \seq (\neg \neg A \imp B) \imp C $} \RightLabel{$\imp_E$}
    \AXC{$A \imp B \seq A \imp B$}
	  \AXC{$ \seq \neg \neg A \imp A$} 
		  \AXC{$\neg \neg A \seq \neg \neg A$}  \RightLabel{$\imp_E$}
	      \BIC{$ \neg \neg A \seq A$} \RightLabel{$\imp_E$}
	\BIC{$ A \imp B, \neg \neg A \seq B$} \RightLabel{$\imp_I$}
	\UIC{$ A \imp B \seq \neg \neg A \imp B$} \RightLabel{$\imp_E$}
  \BIC{$ (\neg \neg A \imp B) \imp C, A \imp B \seq C$} \RightLabel{$\imp_I$}
  \UIC{$ (\neg \neg A \imp B) \imp C \seq (A \imp B) \imp C$}
\end{prooftree}
\end{footnotesize}
\end{frame}



\begin{frame}{Contextual Natural Deduction}
\begin{figure}[h!]
\label{figure:NDd}
\begin{calculus}
\begin{prooftree}
\AXC{$ $} \RightLabel{$axiom$}
\UIC{$ \Gamma, A \seq A$}
\end{prooftree}
\begin{prooftree}
\AXC{$ \Gamma, A \seq \C_{\pi}[B]$} \RightLabel{$\imp_I (\pi)$}
\UIC{$ \Gamma \seq \C_{\pi}[A \imp B]$}
\end{prooftree}
\begin{prooftree}
\AXC{$ \Gamma \seq \C^1_{\pi_1}[A \imp B]$}
		\AXC{$ \Gamma \seq \C^2_{\pi_2}[A]$} \RightLabel{$\imp_E^{\rightharpoonup} (\pi_1;\pi_2)$}
	\BIC{$ \Gamma \seq \C^1_{\pi_1}[\C^2_{\pi_2}[B]]$}
\end{prooftree}
\begin{prooftree}
\AXC{$ \Gamma \seq \C^1_{\pi_1}[A \imp B]$} 
		\AXC{$ \Gamma \seq \C^2_{\pi_2}[A]$}\RightLabel{$\imp_E^{\leftharpoonup} (\pi_1;\pi_2)$}
	\BIC{$ \Gamma \seq \C^2_{\pi_2}[\C^1_{\pi_1}[B]]$}
\end{prooftree}
\begin{center}
\textbf{Note:} $\pi$, $\pi_1$ and $\pi_2$ must be positive positions.
\end{center}
\end{calculus}
\caption{The contextual natural deduction calculus {\NDd}}
\end{figure} 
\end{frame}

\begin{frame}{Comparing {\ND} and {\NDd}}{Double Negation Elimination}

Double negation elimination axiom schema:
$$
dne : \neg \neg F \imp F
$$

Deriving $(A \imp B) \imp C$ from $(\neg \neg A \imp B) \imp C$ in {\NDd}:

\begin{footnotesize}
\begin{prooftree}
\AXC{$ \seq \neg \neg A \imp A$} 
		\AXC{$ (\neg \neg A \imp B) \imp C \seq (\neg \neg A \imp B) \imp C$} \RightLabel{$\imp_E^{\leftharpoonup} (\eS; 11)$}
	\BIC{$ (\neg \neg A \imp B) \imp C  \seq (A \imp B) \imp C$}
\end{prooftree}
\end{footnotesize}

\medskip
\medskip
\medskip
And in {\ND}:

\begin{footnotesize}
\begin{prooftree}
\AXC{$(\neg \neg A \imp B) \imp C \seq (\neg \neg A \imp B) \imp C $} \RightLabel{$\imp_E$}
    \AXC{$A \imp B \seq A \imp B$}
	  \AXC{$ \seq \neg \neg A \imp A$} 
		  \AXC{$\neg \neg A \seq \neg \neg A$}  \RightLabel{$\imp_E$}
	      \BIC{$ \neg \neg A \seq A$} \RightLabel{$\imp_E$}
	\BIC{$ A \imp B, \neg \neg A \seq B$} \RightLabel{$\imp_I$}
	\UIC{$ A \imp B \seq \neg \neg A \imp B$} \RightLabel{$\imp_E$}
  \BIC{$ (\neg \neg A \imp B) \imp C, A \imp B \seq C$} \RightLabel{$\imp_I$}
  \UIC{$ (\neg \neg A \imp B) \imp C \seq (A \imp B) \imp C$}
\end{prooftree}
\end{footnotesize}
\end{frame}


\begin{frame}{Comparing {\ND} and {\NDd}}{Skolemization}

Skolemization axiom schema:
$$
sk : \ex x. F[x] \imp F[f_{sk}(x_1,\ldots,x_n)]
$$
where $x_1,\ldots,x_n$ free-variables of $F$ and $f_{sk}$ new skolem symbol.

\medskip
\medskip

Deriving skolemization $(A \imp B) \imp P(c)$ from $(A \imp B) \imp \ex x. P(x)$ in {\NDd}:

\begin{footnotesize}
\begin{prooftree}
\AXC{$ \seq \ex x. P(x) \imp P(c)$} 
		\AXC{$ (A \imp B) \imp \ex x. P(x) \seq (A \imp B) \imp \ex x. P(x)$} \RightLabel{$\imp_E^{\leftharpoonup} (\eS;0)$}
	\BIC{$  (A \imp B) \imp \ex x. P(x)  \seq  (A \imp B) \imp P(c)$}
\end{prooftree}
\end{footnotesize}

\medskip
\medskip
\medskip
And in {\ND}:

\begin{footnotesize}
\begin{prooftree}
\AXC{$ \seq \ex x. P(x) \imp P(c)$} 
	\AXC{$ \ldots \seq A \imp B$}
			\AXC{$ \ldots \seq (A \imp B) \imp \ex x. P(x) $} \RightLabel{$\imp_E$}
		\BIC{$ (A \imp B) \imp \ex x. P(x), A \imp B \seq \ex x. P(x)$} \RightLabel{$\imp_E$}
	\BIC{$A \imp B, (A \imp B) \imp \ex x. P(x)  \seq P(c)$} \RightLabel{$\imp_I$}
	\UIC{$(A \imp B) \imp \ex x. P(x)  \seq \lambda c^{A \imp B}.  (A \imp B) \imp P(c)$}
\end{prooftree}
\end{footnotesize}
\end{frame}



\begin{frame}{Curry-Howard Isomorphism}{Natural Deduction}
\begin{figure}[h!]
\begin{calculus}
\begin{prooftree}
\AXC{$ $} \RightLabel{$axiom$}
\UIC{$ \Gamma, a: A \seq a: A$}
\end{prooftree}
\begin{prooftree}
\AXC{$ \Gamma, a: A \seq b: B$} \RightLabel{$\imp_I$}
\UIC{$ \Gamma \seq \lambda a^A. b : A \imp B$}
\end{prooftree}
\begin{prooftree}
\AXC{$ \Gamma \seq f: A \imp B $}
		\AXC{$ \Gamma \seq a: A$} \RightLabel{$\imp_E$}
	\BIC{$ \Gamma \seq (f \ a) : B$}
\end{prooftree}
\end{calculus}
\caption{The natural deduction calculus \ND}
\label{figure:ND}
\end{figure}
\end{frame}
 


\begin{frame}{Curry-Howard Isomorphism}{Contextual Natural Deduction {\NDd}}
\begin{figure}[h!]
\label{figure:NDd}
\begin{calculus}
\begin{prooftree}
\AXC{$ $} \RightLabel{$axiom$}
\UIC{$ \Gamma, a: A \seq a: A$}
\end{prooftree}
\begin{prooftree}
\AXC{$ \Gamma, a: A \seq b: \C_{\pi}[B]$} \RightLabel{$\imp_I (\pi)$}
\UIC{$ \Gamma \seq \lambda_{\pi} a^A. b : \C_{\pi}[A \imp B]$}
\end{prooftree}
\begin{prooftree}
\AXC{$ \Gamma \seq f: \C^1_{\pi_1}[A \imp B]$}
		\AXC{$ \Gamma \seq a: \C^2_{\pi_2}[A]$} \RightLabel{$\imp_E^{\rightharpoonup} (\pi_1;\pi_2)$}
	\BIC{$ \Gamma \seq (f \ a)^{\rightharpoonup}_{(\pi_1;\pi_2)} : \C^1_{\pi_1}[\C^2_{\pi_2}[B]]$}
\end{prooftree}
\begin{prooftree}
\AXC{$ \Gamma \seq f: \C^1_{\pi_1}[A \imp B]$} 
		\AXC{$ \Gamma \seq a: \C^2_{\pi_2}[A]$}\RightLabel{$\imp_E^{\leftharpoonup} (\pi_1;\pi_2)$}
	\BIC{$ \Gamma \seq (f \ a)^{\leftharpoonup}_{(\pi_1;\pi_2)} : \C^2_{\pi_2}[\C^1_{\pi_1}[B]]$}
\end{prooftree}
\begin{center}
\textbf{Note:} $\pi$, $\pi_1$ and $\pi_2$ must be positive positions.
\end{center}
\end{calculus}
\caption{The contextual natural deduction calculus {\NDd}}
\end{figure} 
\end{frame}



\begin{frame}{Example}{Double Negation Elimination}

Deriving $(A \imp B) \imp C$ from $(\neg \neg A \imp B) \imp C$ in {\NDd}:

\begin{footnotesize}
\begin{prooftree}
\AXC{$ \seq dne: \neg \neg A \imp A$} 
		\AXC{$ a: \ldots \seq a: (\neg \neg A \imp B) \imp C$} \RightLabel{$\imp_E^{\leftharpoonup} (\eS; 11)$}
	\BIC{$ a: (\neg \neg A \imp B) \imp C  \seq \alert{(dne \ a)^{\leftharpoonup}_{(\eS;11)}} : (A \imp B) \imp C$}
\end{prooftree}
\end{footnotesize}

\medskip
\medskip
\medskip
And in {\ND}:

\begin{footnotesize}
\begin{prooftree}
\AXC{$a: \ldots \seq a: (\neg \neg A \imp B) \imp C $} \RightLabel{$\imp_E$}
    \AXC{$c : \ldots \seq c: A \imp B$}
	  \AXC{$ \seq dne: \neg \neg A \imp A$} 
		  \AXC{$d: \ldots \seq d: \neg \neg A$}  \RightLabel{$\imp_E$}
	      \BIC{$ d: \ldots \seq (dne \ d): A$} \RightLabel{$\imp_E$}
	\BIC{$ d: \ldots, c: \ldots \seq (c \ (dne \ d)): B$} \RightLabel{$\imp_I$}
	\UIC{$ c: \ldots \seq \lambda d^{\neg \neg A}.(c \ (dne \ d)): \neg \neg A \imp B$} \RightLabel{$\imp_E$}
  \BIC{$ a: \ldots, c: \ldots \seq (a \ \lambda d.(c \ (dne \ d))): C$} \RightLabel{$\imp_I$}
  \UIC{$ a: (\neg \neg A \imp B) \imp C \seq \alert{\lambda c^{A \imp B}. (a \ \lambda d^{\neg \neg A}.(c \ (dne \ d)))}: (A \imp B) \imp C$}
\end{prooftree}
\end{footnotesize}
\end{frame}



\begin{frame}{Example}{Skolemization}

Deriving skolemization $(A \imp B) \imp P(c)$ from $(A \imp B) \imp \ex x. P(x)$ in {\NDd}:

\begin{footnotesize}
\begin{prooftree}
\AXC{$ \seq sk: \ex x. P(x) \imp P(c)$} 
		\AXC{$ a: \ldots \seq a: (A \imp B) \imp \ex x. P(x)$} \RightLabel{$\imp_E^{\leftharpoonup} (\eS;0)$}
	\BIC{$ a: (A \imp B) \imp \ex x. P(x)  \seq \alert{(sk \ a)^{\leftharpoonup}_{(\eS;0)}} : (A \imp B) \imp P(c)$}
\end{prooftree}
\end{footnotesize}

\medskip
\medskip
\medskip
And in {\ND}:

\begin{footnotesize}
\begin{prooftree}
\AXC{$ \seq sk: \ex x. P(x) \imp P(c)$} 
	\AXC{$c : \ldots \seq c: A \imp B$}
			\AXC{$a: \ldots \seq a: (A \imp B) \imp \ex x. P(x) $} \RightLabel{$\imp_E$}
		\BIC{$ a : (A \imp B) \imp \ex x. P(x), c: A \imp B \seq (a \ c): \ex x. P(x)$} \RightLabel{$\imp_E$}
	\BIC{$c: A \imp B, a: (A \imp B) \imp \ex x. P(x)  \seq (sk \ (a \ c)) : P(c)$} \RightLabel{$\imp_I$}
	\UIC{$a : (A \imp B) \imp \ex x. P(x)  \seq \alert{\lambda c^{A\imp B}.  (sk \ (a \ c))} : (A \imp B) \imp P(c)$}
\end{prooftree}
\end{footnotesize}
\end{frame}


\begin{frame}{Completeness}

\begin{theorem}[Completeness]
\label{theorem:Completeness}
If $T$ is provable in \ND, then $T$ is provable in \NDd.
\end{theorem}

\begin{definition}[Translation of $\lambda$-terms into $\lambda^d$-terms]
\begin{itemize}
\item $\tN{v} \defEq v$ (for a variable $v$).
%
\item $\tN{\lambda v^T. t} \defEq \lambda_{\eS} v^T. \tN{t}$ 
\item $\tN{(m \ n)} \defEq (\tN{m} \ \tN{n})_{(\eS;\eS)}$
\end{itemize}
\end{definition}
\end{frame}


\begin{frame}{Soundness}

\begin{theorem}[Soundness]
\label{theorem:Soundness}
If $T$ is provable in \NDd, then $T$ is provable in \ND.
\end{theorem}
\end{frame}


\begin{frame}{(Un)Soundness}{Translating $\lambda^d$-terms into $\lambda$-terms}

\begin{itemize}
\item If $t$ is an $(\rightharpoonup)$-application of the form $(f \ a)^{\rightharpoonup}_{(\pi_1;\pi_2)}$, the translation is defined by two successive inductions, firstly on the position $\pi_1$ and then (when $\pi_1 = \eS$) on $\pi_2$, according to the cases below:
	\begin{itemize}
	\item If $\pi_1 = 0\pi$, it is the case that $t$ matches $(f^{C \imp D} \ a)^{\rightharpoonup}_{(0\pi;\pi_2)}$, and then
	$$\tP{t} \defEq \lambda c^C. \tP{((f \ c) \ a))^{\rightharpoonup}_{(\pi;\pi_2)}}$$
	%
	\item If $\pi_1 = 1\pi'$, then there is at least one occurrence of the digit $1$ in $\pi'$, since $\pi_1$ is positive and $\pi'$ is negative. Therefore, $\pi_1$ is necessarily of the form $10\ldots01\pi$ and $t$ matches $(f^{(C_1 \imp \ldots C_n \imp (T_{\pi}[A \imp B] \imp D_1)) \imp D_2} \ a)^{\rightharpoonup}_{(10\ldots01\pi;\pi_2)}$. Then
	\begin{scriptsize}
	$$\tP{t} \defEq \lambda k^{C_1 \imp \ldots C_n \imp (T_{\pi}[B] \imp D_1)}. (f \ \lambda c_1^{C_1} \ldots c_n^{C_n}. \lambda h^{T_{\pi}[A \imp B]}. (k \ c_1 \ldots c_n \ \tP{(h \ a)^{\rightharpoonup}_{(\pi;\pi_2)}})$$
	\end{scriptsize}
	%	
	\item If $\pi_1 = \eS$ and $\pi_2 = 0\pi$, it is the case that $t$ matches $(f \ a^{C \imp D})^{\rightharpoonup}_{(\eS;0\pi)}$, and then
	$$\tP{t} \defEq \lambda c^C. \tP{(f \ (a \ c))^{\rightharpoonup}_{(\eS;\pi)}}$$
	%
	\item  If $\pi_1 = \eS$ and $\pi_2 = 1\pi'$, then there is at least one occurrence of the digit $1$ in $\pi'$, since $\pi_2$ is positive and $\pi'$ is negative. Therefore, $\pi_2$ is of the form $10\ldots01\pi$ and $t$ matches $(f^{A \imp B} \ a^{(C_1 \imp \ldots C_n \imp (T_{\pi}[A] \imp D_1))\imp D_2})^{\rightharpoonup}_{(\eS;10\ldots01\pi)}$. Then
	\begin{scriptsize}
	$$\tP{t} \defEq \lambda k^{C_1 \imp \ldots C_n \imp (T_{\pi}[B] \imp D_1)}. (a \ \lambda c_1^{C_1} \ldots c_n^{C_n}. \lambda h^{T_{\pi}[A]}. (k \ c_1 \ldots c_n \ \tP{(f \ h)^{\rightharpoonup}_{(\eS;\pi)}}))$$
	\end{scriptsize}
	%
	\item If $\pi_1 = \pi_2 = \eS$, it is the case that $t$ matches $(f \ a)^{\rightharpoonup}_{(\eS;\eS)}$, and then
	$$\tP{t} \defEq (\tP{f} \ \tP{a})$$
	\end{itemize}
\end{itemize}
\end{frame}


\begin{frame}{(Un)Soundness}{Translating $\lambda^d$-terms into $\lambda$-terms}

\begin{itemize}
\item If $t$ is an $(\leftharpoonup)$-application of the form $(f \ a)^{\leftharpoonup}_{(\pi_1;\pi_2)}$, the translation is analogous to the previous case for $(f \ a)^{\rightharpoonup}_{(\pi_1;\pi_2)}$, but the induction is made firstly on the position $\pi_2$ and only then (when $\pi_2 = \eS$) on $\pi_1$. For the sake of clarity, all cases are shown below:
	\begin{itemize}
	\item If $\pi_2 = 0\pi$, it is the case that $t$ matches $(f \ a^{C \imp D})^{\leftharpoonup}_{(\pi_1;0\pi)}$, and then
	$$\tP{t} \defEq \lambda c^C. \tP{(f \ (a \ c))^{\leftharpoonup}_{(\pi_1;\pi)}}$$
	%
	\item If $\pi_2 = 1\pi'$, then there is at least one occurrence of the digit $1$ in $\pi'$, since $\pi_2$ is positive and $\pi'$ is negative. Therefore, $\pi_2$ is necessarily of the form $10\ldots01\pi$ and $t$ matches $(f \ a^{(C_1 \imp \ldots C_n \imp (T_{\pi}[A] \imp D_1))\imp D_2})^{\leftharpoonup}_{(\pi_1;10\ldots01\pi)}$. Then
	\begin{scriptsize}
	$$\tP{t} \defEq \lambda k^{C_1 \imp \ldots C_n \imp (T_{\pi}[B] \imp D_1)}. (a \ \lambda c_1^{C_1} \ldots c_n^{C_n}. \lambda h^{T_{\pi}[A]}. (k \ c_1 \ldots c_n \ \tP{(f \ h)^{\leftharpoonup}_{(\pi_1;\pi)}}))$$
	\end{scriptsize}
	%	
	\item If $\pi_2 = \eS$ and $\pi_1 = 0\pi$, it is the case that $t$ matches $(f^{C \imp D} \ a)^{\leftharpoonup}_{(0\pi;\eS)}$, and then
	$$\tP{t} \defEq \lambda c^C. \tP{((f \ c) \ a)^{\leftharpoonup}_{(\pi;\eS)}}$$
	%
	\item  If $\pi_2 = \eS$ and $\pi_1 = 1\pi'$, then there is at least one occurrence of the digit $1$ in $\pi'$, since $\pi_1$ is positive and $\pi'$ is negative. Consequently, $\pi_1$ is of the form $10\ldots01\pi$ and $t$ matches $(f^{(C_1 \imp \ldots C_n \imp (T_{\pi}[A \imp B] \imp D_1)) \imp D_2} \ a)^{\leftharpoonup}_{(10\ldots01\pi;\eS)}$. Then
	\begin{scriptsize}
	$$\tP{t} \defEq \lambda k^{C_1 \imp \ldots C_n \imp (T_{\pi}[B] \imp D_1)}. (f \ \lambda c_1^{C_1} \ldots c_n^{C_n}. \lambda h^{T_{\pi}[A \imp B]}. (k \ c_1 \ldots c_n \ \tP{(h \ a)^{\leftharpoonup}_{(\pi;\eS)}})$$
	\end{scriptsize}
	%
	\item If $\pi_2 = \pi_1 = \eS$, it is the case that $t$ matches $(f \ a)^{\leftharpoonup}_{(\eS;\eS)}$, and then
	$$\tP{t} \defEq (\tP{f} \ \tP{a})$$	
	\end{itemize}
\end{itemize}
\end{frame}


\begin{frame}{(Un)Soundness}{Translating $\lambda^d$-terms into $\lambda$-terms}

\begin{itemize}
\item If $t$ is a variable, then $\tP{t} \defEq t$
%
\item If $t$ is an abstraction of the form $\lambda_{\pi} a^A. b$, the translation is defined by induction on the position $\pi$, according to the cases below:
	\begin{itemize}
	\item If $\pi = 0\pi'$, it is the case that $t$ matches $\lambda_{0\pi'} a^A. b^{C \imp D}$, and then
	$$\tP{t} \defEq \lambda c^C. \tP{\lambda_{\pi'} a^A. (b c)}$$
	%
	\item If $\pi = 1\pi'$, then there is at least one occurrence of the digit $1$ in $\pi'$, since $\pi$ is positive and $\pi'$ is negative. Therefore, $\pi$ is necessarily of the form $10\ldots01\pi''$ and $t$ matches $\lambda_{10\ldots01\pi''} a^A. f^{(C_1 \imp \ldots C_n \imp (T_{\pi''}[B] \imp D_1)) \imp D_2}$. Then
	\begin{scriptsize}
	$$\tP{t} \defEq \lambda k^{C_1 \imp \ldots C_n \imp (T_{\pi''}[A \imp B] \imp D_1)}. (f \ \lambda c_1^{C_1} \ldots c_n^{C_n}. \lambda h^{T_{\pi''}[B]}. (k \ c_1 \ldots c_n \ \tP{\lambda_{\pi''} a^A. h})$$
	\end{scriptsize}
	%	
	\item If $\pi = \eS$, it is the case that $t$ matches $\lambda_{\eS} a. f$, and then
	$$\tP{t} \defEq \lambda a. \tP{f}$$
	\end{itemize}
\end{itemize}

\end{frame}


\begin{frame}{(Un)Soundness}{Translating $\lambda^d$-terms into $\lambda$-terms}

\begin{itemize}
\item If $t$ is a variable, then $\tP{t} \defEq t$
%
\item If $t$ is an abstraction of the form $\lambda_{\pi} a^A. b$, the translation is defined by induction on the position $\pi$, according to the cases below:
	\begin{itemize}
	\item If $\pi = 0\pi'$, it is the case that $t$ matches $\lambda_{0\pi'} a^A. b^{C \imp D}$, and then
	$$\tP{t} \defEq \lambda c^C. \tP{\lambda_{\pi'} a^A. (b c)}$$
	%
	\item If $\pi = 1\pi'$, then there is at least one occurrence of the digit $1$ in $\pi'$, since $\pi$ is positive and $\pi'$ is negative. Therefore, $\pi$ is necessarily of the form $10\ldots01\pi''$ and $t$ matches $\lambda_{10\ldots01\pi''} a^A. f^{(C_1 \imp \ldots C_n \imp (T_{\pi''}[B] \imp D_1)) \imp D_2}$. Then
	\begin{scriptsize}
	\alert{$$\tP{t} \defEq \lambda k^{C_1 \imp \ldots C_n \imp (T_{\pi''}[A \imp B] \imp D_1)}. (f \ \lambda c_1^{C_1} \ldots c_n^{C_n}. \lambda h^{T_{\pi''}[B]}. (k \ c_1 \ldots c_n \ \tP{\lambda_{\pi''} a^A. h})$$}
	\end{scriptsize}
	%	
	\item If $\pi = \eS$, it is the case that $t$ matches $\lambda_{\eS} a. f$, and then
	$$\tP{t} \defEq \lambda a. \tP{f}$$
	\end{itemize}
\end{itemize}

\begin{center}
\alert{
\textbf{Intuitionistic Contextual Soundness Condition:} \\
If $\pi$ contains the digit $1$, then $a$ is not allowed to occur in $f$.
}
\end{center}
\end{frame}


\begin{frame}{From Intuitionistic to Classical Logic}{Proving Peirce's Law and the Double Negation Elimination Principle}

%\begin{footnotesize}
\begin{prooftree}
	\AXC{$ $} \RightLabel{$axiom$}
	\UIC{$p: P, a: (Q \imp P) \seq p: P$} \RightLabel{$\imp_I$}
	\UIC{$p: P \seq \lambda a^{(Q \imp P)}. p : (Q \imp P) \imp P$} \RightLabel{$\imp_I (11)$}
	\UIC{$\seq \lambda_{11} p^P. \lambda a^{(Q \imp P)}. p :  ((P \imp Q) \imp P) \imp P$}
\end{prooftree}
%\end{footnotesize}

\medskip

%\begin{footnotesize}
\begin{prooftree}
	\AXC{$ $} \RightLabel{$axiom$}
	\UIC{$p: P, a: (\bot \imp \bot) \seq p: P$} \RightLabel{$\imp_I$}
	\UIC{$p: P \seq \lambda a^{(\bot \imp \bot)}. p : (\bot \imp \bot) \imp P$} \RightLabel{$\imp_I (11)$}
	\UIC{$\seq \lambda_{11} p^P. \lambda a^{(\bot \imp \bot)}. p :  ((P \imp \bot) \imp \bot) \imp P$}
\end{prooftree}
%\end{footnotesize}

\end{frame}


\begin{frame}{From Intuitionistic to Classical Logic}{Three Ways}

\begin{itemize}
\item Add classical principles as axioms to shallow natural deduction
\item Use a multi-conclusion natural deduction calculus
\item \alert{Allow unrestricted contextual natural deduction inference rules}
\end{itemize}

\end{frame}


\newcommand{\sub}[3]{#1[#2\backslash#3]}
\newcommand{\unfold}{\rightsquigarrow_{\delta}} % unfolding
\newcommand{\br}{\rightsquigarrow_{\beta}} % beta reduction
\newcommand{\unfoldbeta}{\rightsquigarrow_{\beta\delta}}

\begin{frame}{Normalization}

\begin{large}
$$
(\lambda a^A.f \  t') \br \sub{f}{a}{t'}
$$

$$
(\lambda_0 a^A. f \ t')_{(0;\eS)} \rightsquigarrow^{?} \sub{f}{a}{t'}
$$

$$
(\lambda b^B. \lambda a^A. (f \ b) \ t')_{(0;\eS)} \rightsquigarrow^{?} \lambda b^B. (\sub{f}{a}{t'} \ b)
$$
\end{large}
\end{frame}

\begin{frame}{Unfolding}
\begin{figure}[h!]
\begin{calculus}
$$
\infer{\lambda c^C. \lambda_{\pi} a^A. (b c)}
{\lambda_{0\pi} a^A. b^{C \imp D}}
$$

$$
\infer{ \lambda k^{C_1 \imp \ldots C_n \imp (T_{\pi}[A \imp B] \imp D_1)}. (f \ \lambda c_1^{C_1} \ldots c_n^{C_n}. \lambda h^{T_{\pi}[B]}. (k \ c_1 \ldots c_n \ \lambda_{\pi} a^A. h)}
{ \lambda_{10\ldots01\pi} a^A. f^{(C_1 \imp \ldots C_n \imp (T_{\pi}[B] \imp D_1)) \imp D_2} }
$$
\end{calculus}
\caption{Unfolding Contextual Abstractions}
\label{figure:Unfolding}
\end{figure}

\begin{figure}[h!]
\begin{calculus}
$$
\infer{ \lambda c^C. ((f \ c) \ a))^{\rightharpoonup}_{(\pi;\pi_2)} }
{(f^{C \imp D} \ a)^{\rightharpoonup}_{(0\pi;\pi_2)}}
%
\qquad
%
\infer{ \lambda c^C. (f \ (a \ c))^{\leftharpoonup}_{(\pi_1;\pi)} }
{ (f \ a^{C \imp D})^{\leftharpoonup}_{(\pi_1;0\pi)} }
$$

$$
\infer{ \lambda c^C. (f \ (a \ c))^{\rightharpoonup}_{(\eS;\pi)} }
{ (f \ a^{C \imp D})^{\rightharpoonup}_{(\eS;0\pi)} }
%
\qquad
%
\infer{ \lambda c^C. ((f \ c) \ a)^{\leftharpoonup}_{(\pi;\eS)} }
{ (f^{C \imp D} \ a)^{\leftharpoonup}_{(0\pi;\eS)} }
$$
\end{calculus}
\caption{Unfolding Contextual Applications with Position Starting with $0$}
\label{figure:Unfolding}
\end{figure}
\end{frame}

\begin{frame}{Unfolding}
\begin{figure}[h!]
\begin{calculus}
$$
\infer{ \lambda k^{C_1 \imp \ldots C_n \imp (T_{\pi}[B] \imp D_1)}. (f \ \lambda c_1^{C_1} \ldots c_n^{C_n}. \lambda h^{T_{\pi}[A \imp B]}. (k \ c_1 \ldots c_n \ (h \ a)^{\rightharpoonup}_{(\pi;\pi_2)}) }
{(f^{(C_1 \imp \ldots C_n \imp (T_{\pi}[A \imp B] \imp D_1)) \imp D_2} \ a)^{\rightharpoonup}_{(10\ldots01\pi;\pi_2)} }
$$

$$
\infer{ \lambda k^{C_1 \imp \ldots C_n \imp (T_{\pi}[B] \imp D_1)}. (a \ \lambda c_1^{C_1} \ldots c_n^{C_n}. \lambda h^{T_{\pi}[A]}. (k \ c_1 \ldots c_n \ (f \ h)^{\leftharpoonup}_{(\pi_1;\pi)})) }
{ (f \ a^{(C_1 \imp \ldots C_n \imp (T_{\pi}[A] \imp D_1))\imp D_2})^{\leftharpoonup}_{(\pi_1;10\ldots01\pi)} }
$$

$$
\infer{ \lambda k^{C_1 \imp \ldots C_n \imp (T_{\pi}[B] \imp D_1)}. (a \ \lambda c_1^{C_1} \ldots c_n^{C_n}. \lambda h^{T_{\pi}[A]}. (k \ c_1 \ldots c_n \ (f \ h)^{\rightharpoonup}_{(\eS;\pi)})) }
{ (f^{A \imp B} \ a^{(C_1 \imp \ldots C_n \imp (T_{\pi}[A] \imp D_1))\imp D_2})^{\rightharpoonup}_{(\eS;10\ldots01\pi)} }
$$

$$
\infer{ \lambda k^{C_1 \imp \ldots C_n \imp (T_{\pi}[B] \imp D_1)}. (f \ \lambda c_1^{C_1} \ldots c_n^{C_n}. \lambda h^{T_{\pi}[A \imp B]}. (k \ c_1 \ldots c_n \ (h \ a)^{\leftharpoonup}_{(\pi;\eS)}) }
{ (f^{(C_1 \imp \ldots C_n \imp (T_{\pi}[A \imp B] \imp D_1)) \imp D_2} \ a)^{\leftharpoonup}_{(10\ldots01\pi;\eS)} }
$$
\end{calculus}
\caption{Unfolding Contextual Applications with Position Starting with $1$}
\label{figure:Unfolding}
\end{figure}
\end{frame}


\begin{frame}{Unfolding and Beta-Reduction}{Example}
\begin{align*}
(\lambda_0 a^A. (\lambda_0 b^B.h \ a) \ t')_{(0;\eS)} & \unfold && (\lambda b_1^B. \lambda a^A. ((\lambda_0 b^B.h \ a) \ b_1) \ t')_{(0;\eS)}  \\
				  & \unfold &&  \lambda b_2^B.((\lambda b_1^B. \lambda a^A. ((\lambda_0 b^B.h \ a) \ b_1) \ b_2) \ t') \\
				  & \br && \lambda b_2^B. (\lambda a^A. ((\lambda_0 b^B.h \ a) \ b_2) \ t')  \\
				  & \br && \lambda b_2^B. ((\lambda_0 b^B.h \ t') \ b_2)  \\
				  & =_{\eta} && (\lambda_0 b^B.h \ t')
\end{align*} 
\end{frame}

\begin{frame}{Unfolding}{Some Theorems}
\begin{itemize}
\item $\unfold$ is terminating. 
\begin{itemize}
\item For all unfolding rules, the sum of the sizes of all positions decreases.
\end{itemize}
%
\mypause
\item $\unfold$ is locally confluent. 
\begin{itemize}
\item There are no critical pairs.
\end{itemize}
%
\mypause
\item $\unfold$ is confluent. 
%\begin{itemize}
%\item By Newman's Lemma \cite{Newman} applied to Theorems \ref{theorem:Termination} and \ref{theorem:LocalConfluence}.
%\end{itemize}
\end{itemize}
\end{frame}


\begin{frame}{Beta-Unfolding}
\begin{itemize}
\item $\unfoldbeta$ is weakly normalizing.
	\begin{itemize}
	\item Just unfold first and beta-reduce later.
	\end{itemize}
\mypause
\item $\unfoldbeta$ is terminating.
	% \begin{itemize}
	% \item $\br$ is terminating. $\unfold$ is terminating. Furthermore,
	% $\unfold$ has neither \emph{duplicating} nor \emph{collapsing} rules. These would be sufficient conditions for their union $\unfoldbeta$ to be also 					terminating, \alert{if both TRSs were first-order}.
	% \end{itemize}
\mypause
\item $\unfoldbeta$ is confluent.
	% \begin{itemize}
	% \item $\br$ is confluent. $\unfold$ is confluent. Confluence is a modular 
	% property for first-order TRSs, but \alert{not necessarily modular for 
	% higher-order TRSs}. Nevertheless, the \alert{critical-pair lemma} holds for 
	% some kinds of higher-order TRSs\ldots
	% \end{itemize}
\end{itemize}
\end{frame}

\begin{frame}{Quadratic Compressibility}
\begin{center}
In the best cases,\\
\alert{{\NDd}-proofs can be quadratically smaller than smallest {\ND}-proofs.}

\medskip
\medskip

\mypause

There is a sequence of theorems $F_n$ whose \\
smallest {\ND}-proofs $\psi_n$ grow at least quadratically (i.e. $s(\psi_n) \in \Omega(n^2)$),\\ 
while there are \\ 
{\NDd}-proofs $\psi^d_n$ of $F_n$ growing at most linearly (i.e. $s(\psi^d_n) \in O(n)$).
\end{center}
\end{frame}

\begin{frame}{Quadratic Compressibility}{Measuring the Size of Types, Terms and Proofs}
\begin{definition}[Size of a Type]
\begin{itemize}
 \item $s(A) \defEq 1$ (if $A$ is an atomic type)
 \item $s(T_1 \imp T_2) \defEq 1 + s(T_1) + s(T_2)$
\end{itemize}
%\hfill\QED
\end{definition}

\begin{definition}[Size of a $\lambda$-term]
\begin{itemize}
 \item $s(v) \defEq 1$ (if $v$ is a variable)
 \item $s(\lambda v^T. t') \defEq 2 + s(T) + s(t')$
 \item $s((m \ n)) \defEq 1 + s(m) + s(n)$ 
\end{itemize}
%\hfill\QED
\end{definition}

\begin{definition}[Size of a $\lambda^d$-term]
\begin{itemize}
 \item $s(v) \defEq 1$ (if $v$ is a variable)
 \item $s(\lambda_{\pi} v^T. t') \defEq 2 + s(T) + s(t') + s(\pi)$
 \item $s((m \ n)^{\rightharpoonup}_{{\pi_1;\pi_2}}) \defEq 1 + s(m) + s(n) + s(\pi_1) + s(\pi_2)$
  \item $s((m \ n)^{\leftharpoonup}_{{\pi_1;\pi_2}}) \defEq 1 + s(m) + s(n) + s(\pi_1) + s(\pi_2)$
\end{itemize}
%\hfill\QED
\end{definition}

%\begin{definition}
%The \emph{size} $s(\psi)$ of a {\ND} proof $\psi$ is $s(\CurryHoward(\psi))$. The \emph{size} $s(\psi)$ of a {\NDd} proof $\psi$ is $s(\CurryHoward_d(\psi))$. 
%%\hfill\QED
%\end{definition}
\end{frame}

\begin{frame}{Quadratic Compressibility}{Proof}
\noindent
Let $F_n \defEq T^n(A \imp B) \imp ( A \imp T^n(B) )$ where:
\begin{align*}
T^0(F) & \defEq F \\
T^n(F) & \defEq (T^{n-1}(F) \imp D_{2n-1}) \imp D_{2n}
\end{align*}

\smallskip

\mypause

\noindent
Let $\psi^d_n \defEq \CurryHoward_d^{-1}(t^d_n)$ where:
$$
t^d_n \defEq \lambda f^{T^n(A \imp B)}. \lambda a^A. (f \ a)_{(\underbrace{11\ldots 1}_{2n}; \eS)}
$$


\mypause

\noindent
Let $\psi_n \defEq \CurryHoward^{-1}(t_n)$ where:
$$
t_n \defEq \xi(t^d_n)
$$

\mypause

\noindent
Note that $\psi_k$ is a smallest {\ND}-proof of $F_k$. Any {\ND}-proof of $F_k$ must (at least) decompose $F_k$ until the subformulas $A \imp B$ and $A$ are obtained and then apply $A \imp B$ to $A$. $\psi_k$ does exactly this and nothing more.
\end{frame}


\begin{frame}{Quadratic Compressibility}{Proof}
By definition, $s(\psi^d_n) = s(t^d_n)$, and $s(t^d_n)$ is computed below:

\medskip
\begin{small}
\begin{align*}
s(t^d_n) & = s(\lambda f^{T^n(A \imp B)}. \lambda a^A. (f \ a)_{(\underbrace{11\ldots 1}_{2n}; \eS)})  \\
 & = 2 + s(T^n(A \imp B)) + s(\lambda a^A. (f \ a)_{(11\ldots 1; \eS)})  \\
				  & = 2 + (3 + 4n) + s(\lambda a^A. (f \ a)_{(11\ldots 1; \eS)}) \\
				  & = 5 + 4n + (2 + s(A) + s((f \ a)_{(11\ldots 1; \eS)})) \\
				  & = 8 + 4n + s((f \ a)_{(11\ldots 1; \eS)}) \\	
				  & = 8 + 4n + (1 + s(f) + s(a) + s(\underbrace{11\ldots 1}_{2n}) + s(\eS)) \\			  
				  & = 8 + 4n + (3 + 2n + 0) \\			  
				  & = 11 + 6n \\					
\end{align*} 
\end{small}
Therefore, \alert{$s(t^d_n) \in O(n)$}.
\end{frame}


\begin{frame}{Quadratic Compressibility}{Proof}
By definition, $s(\psi_n) = s(t_n)$, and $s(t_n)$ is computed below:

\medskip
\begin{small}
\begin{align*}
s(t_n) & = s(\xi(\lambda f^{T^n(A \imp B)}. \lambda a^A. (f \ a)_{(\underbrace{11\ldots 1}_{2n}; \eS)}))  \\
& = s(\lambda f^{T^n(A \imp B)}. \lambda a^A. \xi((f \ a)_{(11\ldots 1; \eS)}))  \\
& = 2 + s(T^n(A \imp B)) + 2 + s(A) + s(\xi((f \ a)_{(11\ldots 1; \eS)}))  \\
& = 8 + 4n + s(\xi((f \ a)_{(\underbrace{11\ldots 1}_{2n}; \eS)}))  \\
%& \defEq && 8 + 4n + q(n)  
\end{align*} 
\end{small}
\end{frame}


\begin{frame}{Quadratic Compressibility}{Proof}
$
s(t_n) = 8 + 4n + s(\xi((f \ a)_{(\underbrace{11\ldots 1}_{2n}; \eS)}))
$. \ \  Let $q(n) \defEq s(\xi((f \ a)_{(\underbrace{11\ldots 1}_{2n}; \eS)}))$. Then:
\begin{small}
\begin{align*}
q(0) & = s(\xi((f \ a)_{(\eS; \eS)}))  =  3 \\
	   & \\
q(n) & = s(\xi((f \ a)_{(\underbrace{1111\ldots 1}_{2n}; \eS)}))  \\
 & = s( \lambda_k^{T^{n-1}(B) \imp D_{2n-1}}.(f \ \lambda h^{T^{n-1}(A\imp B)}.\xi((h \ a)_{(\underbrace{11\ldots 1}_{2n-2}; \eS)})))  \\
 & = 2 + s(T^{n-1}(B) \imp D_{2n-1}) + s((f \ \lambda h^{T^{n-1}(A\imp B)}.\xi((h \ a)_{(11\ldots 1; \eS)})))  \\
 & = 2 + 4(n-1) + 3 + s((f \ \lambda h^{T^{n-1}(A\imp B)}.\xi((h \ a)_{(11\ldots 1; \eS)})))  \\
 & = 1 + 4n + s((f \ \lambda h^{T^{n-1}(A\imp B)}.\xi((h \ a)_{(11\ldots 1; \eS)})))  \\
 & = 1 + 4n + 2 + s(\lambda h^{T^{n-1}(A\imp B)}.\xi((h \ a)_{(11\ldots 1; \eS)}))  \\
 & = 3 + 4n + s(\lambda h^{T^{n-1}(A\imp B)}.\xi((h \ a)_{(11\ldots 1; \eS)}))  \\
 & = 5 + 4n + s(T^{n-1}(A\imp B)) + s(\xi((h \ a)_{(\underbrace{11\ldots 1}_{2n-2}; \eS)})) = 4 + 8n + q(n-1) 
% & = 5 + 4n + 4(n-1) + 3 + q(n-1)  \\
% & = 4 + 8n + q(n-1)  \\
\end{align*} 
\end{small}
\end{frame}


\begin{frame}{Quadratic Compressibility}{Proof}
$$
s(t_n) = 8 + 4n + q(n)
$$

\medskip

\begin{align*}
q(0) & =   3 \\
	   & \\
q(n) & =  4 + 8n + q(n-1)
\end{align*} 

\medskip
\medskip
%\noindent
Solving the recurrence relation above gives the following closed-form for $q$:
$$
q(n) = 4n^2 + 8n + 3
$$
%

\medskip
\noindent
Therefore, \alert{$s(\psi_n) \in \Omega(n^2)$}.
\end{frame}

% \begin{frame}{The Proof Compressibility and Compression Problems}
% \begin{itemize}
% \item The compressibility decision problem:
% \begin{itemize}
% 	\item \textbf{input:} a theorem $T$ and a {\ND}({\bf d})-proof $\psi$ of $T$.
% 	\item \textbf{output:} 
% 	\begin{itemize}
% 		\item \emph{yes}, if there is an {\NDd}-proof $d(\psi)$ of $T$ s.t. $s(d(\psi)) < s(\psi)$.
% 		\item \emph{no}, otherwise.
% 	\end{itemize}
% 	\mypause
% 	\item Decidable
% 	\mypause
% 	\item Complexity: in $NP$
% \end{itemize}
% \mypause
% \item The compression function problem:
% \begin{itemize}
% 	\item \textbf{input:} a theorem $T$ and a {\ND}({\bf d})-proof $\psi$ of $T$.
% 	\item \textbf{output:} an {\NDd}-proof $d(\psi)$ of $T$ s.t. 
% 	\begin{itemize}
% 		\item for any {\NDd}-proof $\psi'$ of $T$, $s(d(\psi)) \leq s(\psi')$.
% 	\end{itemize}
% 	\mypause
% 	\item Computable
% %	\item Complexity: in $FNP^{NP}$
% \end{itemize}
% \end{itemize}
% \end{frame}

% \begin{frame}{The Proof Compressibility and Compression Problems}
% \begin{itemize}
% \item The compressibility decision problem \alert{(restricted to folding)}:
% \begin{itemize}
% 	\item \textbf{input:} a theorem $T$ and a {\ND}({\bf d})-proof $\psi$ of $T$.
% 	\item \textbf{output:} 
% 	\begin{itemize}
% 		\item \emph{yes}, if there is an {\NDd}-proof $d(\psi)$ of $T$ s.t. $s(d(\psi)) < s(\psi)$ and \alert{$d(\psi) \unfold \psi$}.
% 		\item \emph{no}, otherwise.
% 	\end{itemize}
% \end{itemize}
% \item The compression function problem \alert{(restricted to folding)}:
% \begin{itemize}
% 	\item \textbf{input:} a theorem $T$ and a {\ND}({\bf d})-proof $\psi$ of $T$.
% 	\item \textbf{output:} an {\NDd}-proof $d(\psi)$ of $T$ s.t. 
% 	\begin{itemize}
% 		\item for any {\NDd}-proof $\psi'$ of $T$, $s(d(\psi)) \leq s(\psi')$.
% 		\item \alert{$d(\psi) \unfold \psi$}
% 	\end{itemize}
% \end{itemize}
% \end{itemize}
% \end{frame}

\begin{frame}{Proof Compression by Folding}
\begin{itemize}
\item $\unfold^{-1}$ is terminating
\begin{itemize}
	\item The term size decreases with every inverse rewriting step.
\end{itemize}
\mypause
\item $\unfold^{-1}$ is not confluent
\begin{itemize}
	\item Let $f: A\imp B$ and $a: (A \imp D) \imp E$. Then: 
	\begin{itemize}
		\item $\lambda k^{B \imp D}. (a \ \lambda h^A.(k \ (f \ h))) \quad \unfold^{-1} \quad (f \ a)_{(\eS ; 11)}$
		\item $\lambda k^{B \imp D}. (a \ \lambda h^A.(k \ (f \ h))) \quad \unfold^{-1} \quad \lambda k^{B \imp D}. (a \ (k \ f)_{(\eS ; 0)})$
	\end{itemize}
\end{itemize}
\end{itemize}
\end{frame}


\begin{frame}{Just Folding is not Enough!}
$$
h_{ab}: A \imp B \qquad h_{bc} : B \imp C \qquad h_{ade} : (A \imp D) \imp E
$$

\medskip
\medskip

$$
\lambda h_{cd}^{C \imp D}. (h_{ade} \ \lambda h_a^A.(h_{cd} \ (h_{bc} \ (h_{ab} \ h_a))))  :  (C \imp D) \imp E
$$

\medskip

\mypause

\begin{center}
\alert{Normal form w.r.t. $\unfold^{-1}$.}

\mypause

Yet, there are smaller $\lambda^d$-terms:
\end{center}

$$
(h_{bc} \ (h_{ab} \ h_{ade})_{(\eS ; 11)})_{(\eS ; 11)}  % :  (C \imp D) \imp E
$$

$$
((h_{bc} \ h_{ab})_{(\eS ; 0)} \ h_{ade})_{(\eS ; 11)}  %:  (C \imp D) \imp E
$$

\medskip
\medskip

\mypause

\begin{center}
To obtain them, we need \alert{folding + beta expansion}
\end{center}
\end{frame}



\begin{frame}{Proof Compression by Folding and Beta Expansion}

$$
h_{ab}: A \imp B \qquad h_{bc} : B \imp C \qquad h_{ade} : (A \imp D) \imp E
$$

\medskip
\medskip

%\begin{small}
\begin{align*}
t & \defEq \lambda h_{cd}^{C \imp D}. (h_{ade} \ \lambda h_a^A.(h_{cd} \ (h_{bc} \ (h_{ab} \ h_a)))) \\
& \br^{-1}  \lambda h_{cd}^{C \imp D}. (h_{ade} \ \lambda h_a^A.(\alert{\lambda k_b^B}. (h_{cd} \ (h_{bc} \ \alert{k_b}))  \ (h_{ab} \ h_a))) \\
& \br^{-1}  \lambda h_{cd}^{C \imp D}. (\alert{\lambda k_{bd}^{B \imp D}}. (h_{ade}  \ \lambda h_a^A.(\alert{k_{bd}}  \ (h_{ab} \ h_a))) \ \alert{\lambda k_b^B}. (h_{cd} \ (h_{bc} \ \alert{k_b}))) \\
& \unfold^{-1}  (h_{bc} \ \alert{\lambda k_{bd}^{B \imp D}}. (h_{ade}  \ \lambda h_a^A.(\alert{k_{bd}}  \ (h_{ab} \ h_a))))_{(\eS ; 11)} \\
& \unfold^{-1}  (h_{bc} \ (h_{ab} \ h_{ade})_{(\eS ; 11)})_{(\eS ; 11)} 
\end{align*}
%\end{small}

\mypause

%\begin{small}
\begin{align*}
t & \defEq \lambda h_{cd}^{C \imp D}. (h_{ade} \ \lambda h_a^A.(h_{cd} \ (h_{bc} \ (h_{ab} \ h_a)))) \\
& \br^{-1} \lambda h_{cd}^{C \imp D}. (h_{ade} \ \lambda h_a^A.(h_{cd} \ (\alert{\lambda k_a^A}. (h_{bc} \ (h_{ab} \ \alert{k_a})) \ h_a))) \\
& \unfold^{-1}  (\alert{\lambda k_a^A}.(h_{bc} \ (h_{ab} \ \alert{k_a})) \ h_{ade})_{(\eS ; 11)} \\
& \unfold^{-1}  ((h_{bc} \ h_{ab})_{(\eS ; 0)} \ h_{ade})_{(\eS ; 11)} 
\end{align*}
%\end{small}
\end{frame}


\begin{frame}{Proof Compression by Folding and Beta Expansion}
\begin{itemize}
\item $\unfoldbeta^{-1}$ is not terminating
\begin{itemize}
	\item Because beta expansion is not terminating
\end{itemize}
\mypause
\item $\unfoldbeta^{-1}$ is not confluent
\begin{itemize}
	\item By the examples in the previous slide
\end{itemize}
\end{itemize}
\end{frame}


% \begin{frame}{Preliminary Implementation Remarks}

% \begin{itemize}
% \item The calculi and a simple generic theorem prover were implemented in ResK

% \item A few hundreds of small and quite shallow theorems were generated

% \item Contextual proofs were in average 5\% smaller than shallow proofs

% \item Smaller prrofs are expected for deeper theorems

% \item But deeper theorems need more efficient computation of candidate positions
% \end{itemize}

% \end{frame}


% \begin{frame}{Other Deep Inference Systems}{\ND}


% \begin{scriptsize}
% \begin{prooftree}
% \AXC{$ (B \imp C) \seq (B \imp C)$}
%     \AXC{$ (A \imp B) \seq (A \imp B)$} 
%     		\AXC{$A \seq A$} \RightLabel{$\imp_E$}
% 	    \BIC{$ (A \imp B), A \seq B$} \RightLabel{$\imp_E$}
%   \BIC{$(A \imp B), (B \imp C), A \seq C$} \RightLabel{$\imp_I$}
%   \UIC{$(A \imp B), (B \imp C) \seq (A \imp C)$} \RightLabel{$\imp_I$}
%   \UIC{$(A \imp B) \seq  (B \imp C) \imp (A \imp C)$} \RightLabel{$\imp_I$}
%   \UIC{$\seq (A \imp B) \imp ((B \imp C) \imp (A \imp C))$}
% \end{prooftree}
% \end{scriptsize}


% $$
% \lambda h^{A\imp B}. \lambda k^{B \imp C}. \lambda a^{A}. (k \ (h \ a))
% $$

% \end{frame}

% \begin{frame}{Other Deep Inference Systems}{\NDd}

% \begin{scriptsize}
% \begin{prooftree}
% \AXC{$ (B \imp C) \seq (B \imp C)$}
%     \AXC{$ (A \imp B) \seq (A \imp B)$} \RightLabel{$\imp_E (\eS;0)$}
%   \BIC{$(A \imp B), (B \imp C) \seq (A \imp C)$} \RightLabel{$\imp_I$}
%   \UIC{$(A \imp B) \seq  (B \imp C) \imp (A \imp C)$} \RightLabel{$\imp_I$}
%   \UIC{$\seq (A \imp B) \imp ((B \imp C) \imp (A \imp C))$}
% \end{prooftree}
% \end{scriptsize}

% And the proof above can be represented by the $\lambda^c$-term: 
% $$
% \lambda h^{A\imp B}. \lambda k^{B \imp C}. (k \ h)_{(\eS;0)}
% $$

% \end{frame}

% \begin{frame}{Other Deep Inference Systems}{Tiu's}

% \newcommand{\la}{\langle}
% \newcommand{\ra}{\rangle}
% \begin{scriptsize}
% \begin{prooftree}
%   \AXC{$\top $} \RightLabel{$i$}
%   \UIC{$\la \la A; C \ra ; \la A; C \ra \ra $} \RightLabel{$=_{unit}$}
%   \UIC{$\la \la A; \la \top ; C \ra \ra ; \la A; C \ra \ra$} \RightLabel{$i$}
%   \UIC{$\la \la A; \la \la B ; B \ra ; C \ra \ra ; \la A; C \ra \ra$} \RightLabel{$s i c \uparrow$}
%   \UIC{$\la \la A; \la ( B , \la B ; C \ra ) ; \la A; C \ra \ra$} \RightLabel{$s c \uparrow$}
%   \UIC{$\la (\la A; B \ra , \la B ; C \ra ) ; \la A; C \ra \ra$} \RightLabel{$=_{currying}$}
%   \UIC{$\la \la A; B \ra ; \la \la B ; C \ra  ; \la A; C \ra \ra \ra$}
% \end{prooftree}
% \end{scriptsize}

% \end{frame}

% \begin{frame}{Other Deep Inference Systems}{Bruennler-McKinley's}

% \begin{scriptsize}
% \begin{prooftree}
%   \AXC{$\top $} \RightLabel{$i$}
%   \UIC{$(A\imp B) \imp (\top \wedge (A \imp B))$} \RightLabel{$w_2$}
%   \UIC{$(A\imp B) \imp (A \imp B)$} \RightLabel{$i$}
%   \UIC{$(A\imp B) \imp ((B \imp C) \imp ((A \imp B) \wedge (B \imp C)))$} \RightLabel{$c$}
%   \UIC{$(A\imp B) \imp ((B \imp C) \imp (((A \imp B) \wedge (B \imp C)) \wedge ((A \imp B) \wedge (B \imp C))))$} \RightLabel{$w_2$}
%   \UIC{$(A\imp B) \imp ((B \imp C) \imp ((B \imp C) \wedge ((A \imp B) \wedge (B \imp C))))$} \RightLabel{$w_1$}
%   \UIC{$(A\imp B) \imp ((B \imp C) \imp ((B \imp C) \wedge (A \imp B)))$} \RightLabel{$i$}
%   \UIC{$(A\imp B) \imp ((B \imp C) \imp (A \imp   (((B \imp C) \wedge (A \imp B)) \wedge A)   ) )$} \RightLabel{$c$}
%   \UIC{$(A\imp B) \imp ((B \imp C) \imp (A \imp  ( (((B \imp C) \wedge (A \imp B)) \wedge A) \wedge (((B \imp C) \wedge (A \imp B)) \wedge A) )  ) )$} \RightLabel{$w_2$}
%   \UIC{$(A\imp B) \imp ((B \imp C) \imp (A \imp  ( (((B \imp C) \wedge (A \imp B)) \wedge A) \wedge ( (A \imp B) \wedge A) )  ) )$} \RightLabel{$w_1$}
%   \UIC{$(A\imp B) \imp ((B \imp C) \imp (A \imp  ( ((B \imp C) \wedge A) \wedge ( (A \imp B) \wedge A) )  ) )$} \RightLabel{$w_1$}
%   \UIC{$(A\imp B) \imp ((B \imp C) \imp (A \imp  ( (B \imp C) \wedge ( (A \imp B) \wedge A) )  ) )$} \RightLabel{$e$}
%   \UIC{$(A\imp B) \imp ((B \imp C) \imp (A \imp  ( (B \imp C) \wedge  B  )  ) )$} \RightLabel{$e$}
%   \UIC{$(A\imp B) \imp ((B \imp C) \imp (A \imp  C)   ) $}
% \end{prooftree}
% \end{scriptsize}

% \begin{scriptsize}
% $$
% i^{A \imp B}.(id \imp (w_2. i^{B \imp C}.(id \imp (c.(w_2 \wedge w_1).i^A .(id \imp (c.( ((w_1 \wedge id). w_1 ) \wedge (w_2 \wedge id) ).(id \wedge e).e))))))
% $$
% \end{scriptsize}

% \end{frame}

% \begin{frame}{Other Deep Inference Systems}{Gundersen's}

% \begin{scriptsize}
% \begin{prooftree}
%   \AXC{$\top $} \RightLabel{$i$}
%   \UIC{$(A\imp B) \imp (\top \wedge (A \imp B))$} \RightLabel{$\equiv$}
%   \UIC{$(A\imp B) \imp (A \imp B)$} \RightLabel{$i$}
%   \UIC{$(A\imp B) \imp ((B \imp C) \imp ((B \imp C) \wedge (A \imp B)))$} \RightLabel{$i$}
%   \UIC{$(A\imp B) \imp ((B \imp C) \imp (A \imp  ( (B \imp C) \wedge  (A \imp B) \wedge A )  ) )$} \RightLabel{$e$}
%   \UIC{$(A\imp B) \imp ((B \imp C) \imp (A \imp  ( (B \imp C) \wedge  B  )  ) )$} \RightLabel{$e$}
%   \UIC{$(A\imp B) \imp ((B \imp C) \imp (A \imp  C)   ) $}
% \end{prooftree}
% \end{scriptsize}

% $$
% \lambda h^{A\imp B}. \lambda k^{B \imp C}. \lambda a^{A}. (k) \ (h) \ a
% $$
% \end{frame}

% \begin{frame}{Other Deep Inference Systems}{Guenot's}

% \begin{scriptsize}
% \begin{prooftree}
%   \AXC{$\top $} \RightLabel{$i$}
%   \UIC{$(\top \imp (A \imp C)) \imp (A \imp C)$} \RightLabel{$\equiv_u$}
%   \UIC{$(A \imp C) \imp (A \imp C)$} \RightLabel{$i$}
%   \UIC{$(((A \imp B) \imp B) \imp C) \imp (A \imp C)$} \RightLabel{$s$}
%   \UIC{$(A \imp B) \imp ((B \imp C) \imp (A \imp C))$} 
% \end{prooftree}
% \end{scriptsize}
% \end{frame}

\begin{frame}{Propositional Resolution}


\end{frame}

\begin{frame}{Current Work}

\begin{itemize}
\item Contextual Natural Deduction:
	\begin{itemize}
	\item go beyond the implicational fragment
	\mypause
	\item investigate beta-expansion / cut-introduction
	\mypause
	\item implement and evaluate compressibility in practice
	\mypause
	\item obtain a syntactic proof of soundness for the classical case
	\mypause
	\item investigate algorithmic interpretations for the classical case
	%\item implement deep-apply and deep-intro tactics in Coq
	\end{itemize}
\item Propositional Resolution:
	\begin{itemize}
	\item develop efficient subsumption algorithms
  \mypause
	\item improve lowering of subproofs
  \mypause
	\item improve split
	\end{itemize}
\end{itemize}
\end{frame}


\begin{frame}{The End}
\begin{itemize}
\item Thanks!
\item Some announcements:
	\begin{itemize}
	\item LowerUnivalents: SMT2013, Helsinki, 8th of July 15:30
	\item Proof Compression Workshop: \\
	16th of September, affiliated with Tableaux, Nancy, France
	\end{itemize}
\item Questions? Comments? Suggestions?
\item www.logic.at/people/bruno/
\end{itemize}

%\begin{footnotesize}
% \begin{itemize}
%\item Deep Natural Deduction 
%\end{itemize}
%\end{footnotesize}


\end{frame}


\end{document}
