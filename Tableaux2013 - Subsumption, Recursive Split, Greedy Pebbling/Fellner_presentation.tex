\documentclass{beamer}

\usepackage[latin1]{inputenc}
\usepackage[british]{babel}
\usepackage{amssymb}
%\usepackage{utf8}
%\usepackage{latexsym}
%\usepackage{rotate}
\usepackage{tikz}
\usepackage{datetime}
\usepackage{bussproofs}

\newdate{date}{16}{09}{2013}
\date{\displaydate{date}}

%\definecolor{vertmoyen}{HTML}{A9CB60}
%\definecolor{vertclair}{HTML}{D3F192}
\definecolor{vertfonce}{HTML}{347003}
%\definecolor{deletecol}{HTML}{3B8A13}
%\definecolor{addcolor} {HTML}{9E0270}

\usecolortheme[named=vertfonce]{structure}

\newcommand{\dual}[1]{\ensuremath{\bar{#1}}}

\newenvironment<>{subpart}[1]
{ \begin{block}#2{#1}
  \begin{itemize}
}{
  \end{itemize}
  \end{block}
}

\newcommand{\recalltoc}{\frame{\tableofcontents[sectionstyle=show/shaded,subsectionstyle=show/show/hide]}}

\newcommand{\Resolution}[2]{\RightLabel{\footnotesize{$#1$}} \BinaryInfC{$#2$}}

%\usepackage[vlined]{algorithm2e}
\usetikzlibrary{arrows}
\usetikzlibrary{positioning}

\tikzstyle{proof edge}=[thick,cap=round]
\tikzstyle{deleted edge}=[proof edge, dashed, color=deletecol]

\newcommand{\proofnode}[3][]{
  \node [anchor=mid, #1] (#2) {#3}
}

\newcommand{\rootnode}[1][]{
  \proofnode[#1]{root}{$\bot$}
}

\newcommand<>{\edgewithlabel}[3]{
  \draw#4 [proof edge, color=vertfonce] (#1) -- (#2) node [above, pos=0.3] {\footnotesize #3}
}

\newcommand<>{\drawchildren}[3]{
  \draw#4 [proof edge] (#1) -- (#2);
  \draw#4 [proof edge] (#1) -- (#3)
}

\newcommand{\addchildren}[5]{
  \proofnode[above left  of=#1]{#2}{#3};
  \proofnode[above right of=#1]{#4}{#5}
}

\newcommand{\withchildren}[5]{
  \addchildren{#1}{#2}{#3}{#4}{#5};
  \drawchildren{#1}{#2}{#4}
}

\newcommand{\withchildrenPivot}[6]{
  \addchildren{#1}{#2}{#3}{#4}{#5};
  \drawchildren{#1}{#2}{#4}
}

\newcommand<>{\crossnode}[2][]{
  \draw#3 [color=red,thick,cap=round,#1] (#2.mid) ++(10:0.3) -- ++(190:0.6);
}

\newcommand<>{\marknode}[2][.33cm]{
  \draw#3 [color=red,line width=1pt] (#2) circle (#1);
}

\newcommand<>{\whitenode}[2][.19cm]{
  \draw#3 [color=black,fill=white,line width=0.5pt] (#2) circle (#1);
}

\newcommand<>{\waitingnode}[2][.19cm]{
  \draw#3 [color=black,fill=vertfonce,fill opacity=0.6,line width=0.5pt] (#2) circle (#1);
}

\newcommand<>{\blacknode}[2][.19cm]{
  \draw#3 [color=black,fill=black,line width=0.5pt] (#2) circle (#1);
}

\newcommand{\proofnodeBW}[2][]{
  \node [anchor=mid, #1] (#2) {};
	\whitenode{#2};
}

\newcommand{\proofnodeBWHidden}[2][]{
  \node [anchor=mid, #1] (#2) {};
}


\newcommand{\rootnodeBW}[1][]{
  \proofnodeBW[#1]{root}{};
}

\newcommand{\addchildrenBW}[3]{
  \proofnodeBW[above left  of=#1]{#2};
  \proofnodeBW[above right of=#1]{#3};
}

\newcommand{\withchildrenBW}[3]{
  \addchildrenBW{#1}{#2}{#3};
  \drawchildren{#1}{#2}{#3};
}


\newcommand{\firstNode}[3] {
	\node [anchor=mid] (#1) at (0,0) {#2};
	\node [color=red] (n#1) at (#1.north) {\tiny{#3}};
}

\newcommand{\nextNode}[4] {
	\node [anchor=mid, right of=#1] (#2) {#3};
	\node [color=red] (n#2) at (#2.north) {\tiny{#4}};
}


\title{Subsumption, Recursive Split and\\Greedy Pebbling}% for Propositional Proof Compression}
%\title{Three algorithms for Propositional Proof Compression}

%\author{Andreas Fellner}
\author[AFellner]{Andreas Fellner\\[2mm]Joseph Boudou, Bruno Woltzenlogel Paleo}
\vspace{2cm}
\date[APPC 2013]{Third Workshop of the Amadeus Project on Proof Compression\\\displaydate{date}}

\begin{document}

\maketitle
%\begin{frame}
%\frametitle{Table of content}
%\begin{itemize}
%\item Subsumption
%\item Recursive Split
%\item Greedy Pebbling
%\end{itemize}
%\end{frame}

\begin{frame}{Overview}
\tableofcontents
\end{frame}

\section{Introduction}

\begin{frame}

\frametitle{Propopositional Resolution Calculus}

\begin{subpart}{Literal}
 \item Variable $v$ or negated variable $\dual{v}$
\end{subpart}

\begin{subpart}{Clause}
	\item Disjunction of literals
	\item Represented as a set
\end{subpart}

\begin{block}{Resolution rule}
	\begin{prooftree}%
		\AxiomC{$\Gamma \vee v$}%
		\AxiomC{$\dual{v} \vee \Delta$}%
		\Resolution{v}{\Gamma \vee \Delta}%
	\end{prooftree}
\end{block}

\end{frame}

\begin{frame}

\begin{block}{Proof as a directed acyclic graph (DAG)}
	\begin{itemize}
		\item node, conclusion, pivot, premise, child
	\end{itemize}
		\centering
		\begin{tikzpicture}[node distance=1.2cm]
			\rootnode;
			\withchildren{root} {n7}{$b$}  {n9}{$\dual{b}$};
			\withchildren{n7}   {n1}{$\dual{a},b$} {n6}{$a$};
			\proofnode[above right of=n9] {n8} {$\dual{b},\dual{a}$};
			\drawchildren{n9} {n6} {n8};
			\withchildren{n6} {n4}{$\dual{c}$} {n5}{a,c};
			\withchildren{n4} {n2}{$\dual{a},\dual{c}$}{n3}{$a,\dual{c}$};
			\uncover<2>{
			\node [color=red] (numb1) at (n1.west) {\tiny{1}};
			\node [color=red] (numb2) at (n2.west) {\tiny{2}};
			\node [color=red] (numb3) at (n3.west) {\tiny{3}};
			\node [color=red] (numb4) at (n4.west) {\tiny{4}};
			\node [color=red] (numb5) at (n5.west) {\tiny{5}};
			\node [color=red] (numb6) at (n6.west) {\tiny{6}};
			\node [color=red] (numb7) at (n7.west) {\tiny{7}};
			\node [color=red] (numb8) at (n8.west) {\tiny{8}};
			\node [color=red] (numb9) at (n9.west) {\tiny{9}};
			\node [color=red] (numb10) at (root.west) {\tiny{10}};
			}
		\end{tikzpicture}
\end{block}
\vspace{-0.5cm}
\begin{block}{Proof as sequence}
	\begin{itemize}
		\item Use topological order
	\end{itemize}
	\begin{tikzpicture}[node distance=0.8cm]
		\usetikzlibrary{calc}
		\firstNode{n1} {$\dual{a},b$} {\uncover<2>{1}};
		\nextNode{n1} {n2} {$\dual{a},\dual{c}$} {\uncover<2>{2}};
		\nextNode{n2} {n3}{$a,\dual{c}$} {\uncover<2>{3}};
		\nextNode{n3} {n4}{$\dual{c}$} {\uncover<2>{4}};
		\nextNode{n4} {n5}{$a,c$}{\uncover<2>{5}};
		\nextNode{n5} {n6}{$a$}{\uncover<2>{6}};
		\nextNode{n6} {n7}{$b$}{\uncover<2>{7}};
		\nextNode{n7} {n8}{$\dual{b},\dual{a}$} {\uncover<2>{8}};
		\nextNode{n8} {n9}{$\dual{b}$} {\uncover<2>{9}};
		\nextNode{n9} {root}{$\bot$} {\uncover<2>{10}};
		\draw [-triangle 45,color=black,thick] (3,-0.8) -- node[auto,text=vertfonce,above]{top-down} (6,-0.8);
		\draw [-triangle 45,color=black,thick] (6,-1.1) -- node[auto,text=vertfonce,below]{bottom-up} (3,-1.1); 
	\end{tikzpicture}
\end{block}
\end{frame}

\section{Subsumption algorithms}
\recalltoc
\begin{frame}
	\frametitle{Subsumption for Proof Compression}
	\begin{subpart}{Idea}
		\item Subsumption
		\begin{itemize}
			\item $C_1$ subsumes $C_2$ iff $C_1 \subset C_2$
		\end{itemize}
		\item Replace subsumed clauses by their subsumers
		\item Fix nodes with changed premises
		\begin{itemize}
			\item Pivot in both premises $\rightarrow$ resolve premises
			\item Pivot missing in a premise $\rightarrow$ use this premise
		\end{itemize}
	\end{subpart}
	\begin{subpart}{Performance}
		\item Worst case quadratic runtime in the proof size
		\item Use literal-tree structure to store visited nodes and check subsumption
	\end{subpart}
\end{frame}

\begin{frame}

	\frametitle{Top-down vs Bottom-up}

	\begin{block}{Top-down}
		\begin{tikzpicture}
			%\nextNode{(0,0)} {p1} {$p_1$} {};
			\firstNode{p1}{$C_1$}{};
			\nextNode{p1} {p2} {$C_2$} {};
			\nextNode{p2} {p3} {$C_3$} {};
			\nextNode{p3} {dots1} {$\ldots$} {};
			\nextNode{dots1} {pk} {$C_k$} {};
			\nextNode{pk} {dots2}{$\ldots$} {};
			\nextNode{dots2} {pn2}{$C_{n-2}$} {};
			\nextNode{pn2} {pn1}{$C_{n-1}$} {};
			\nextNode{pn1} {pn}{$C_n$} {};
			\draw (pk) to [out = 90, in = 90] node[above,pos = 0.8] {$\subset$} (p1);
			\draw (pk) to [out = 90, in = 90] node[above, pos = 0.8] {$\subset$} (p2); %{$\overset{\tiny{?}}{\tiny{<}}$} (p2);
			\draw (pk) to [out = 90, in = 90] node[above, pos = 0.8] {$\subset$} (p3);
			\draw [color=vertfonce,thick,-triangle 45] (3,-0.6) -- (6,-0.6);
			%\draw [->] (p1){}+(0,-1) -- (pn){}+(0,-1);
			%\draw [color=blue] (p1) circle (1);
			%\draw [color=blue] (pn) circle (1);
		\end{tikzpicture}
	\end{block}
		\begin{block}{Bottom-up}
		\begin{tikzpicture}
			%\nextNode{(0,0)} {p1} {$p_1$} {};
			\firstNode{p1}{$c_1$}{};
			\nextNode{p1} {p2} {$C_2$} {};
			\nextNode{p2} {p3} {$C_3$} {};
			\nextNode{p3} {dots1} {$\ldots$} {};
			\nextNode{dots1} {pk} {$C_k$} {};
			\nextNode{pk} {dots2}{$\ldots$} {};
			\nextNode{dots2} {pn2}{$C_{n-2}$} {};
			\nextNode{pn2} {pn1}{$C_{n-1}$} {};
			\nextNode{pn1} {pn}{$C_n$} {};
			\draw (pk) to [out = 90, in = 90] node[above,pos = 0.8] {$\supset^*$} (pn2);
			\draw (pk) to [out = 90, in = 90] node[above, pos = 0.8] {$\supset^*$} (pn1); %{$\overset{\tiny{?}}{\tiny{<}}$} (p2);
			\draw (pk) to [out = 90, in = 90] node[above, pos = 0.8] {$\supset^*$} (pn);
			\draw [color=vertfonce,thick,-triangle 45] (6,-0.6) -- (3,-0.6);
		\end{tikzpicture}
	\end{block}
	$C \subset^* D$ iff $C \subset D$ and $C$ is not an ancestor of $D$
\end{frame}

\begin{frame}
	\frametitle{Top-down Subsumption Example}
	\centering
	\begin{tikzpicture}[node distance=1.2cm]
		\rootnode;
		\withchildren{root} {n5}{$a$}  {n6}{$\dual{a}$};
		\addchildren{n5}   {n3}{$\dual{b}$} {n4}{$a,b$};
		\drawchildren<-2>{n5} {n3} {n4};
		\withchildren{n3} {n1}{$\dual{a},\dual{b}$} {n2}{$a$};
		\only<1>{
		\node [color=red] (numb1) at (n1.west) {\tiny{1}};
		\node [color=red] (numb2) at (n2.west) {\tiny{2}};
		\node [color=red] (numb3) at (n3.west) {\tiny{3}};
		\node [color=red] (numb4) at (n4.west) {\tiny{4}};
		\node [color=red] (numb5) at (n5.west) {\tiny{5}};
		\node [color=red] (numb6) at (n6.west) {\tiny{6}};
		\node [color=red] (numb7) at (root.west) {\tiny{7}};
		}
		\marknode<2>{n2};
		\marknode<2>{n4};
		\crossnode<3->{n4};
		\drawchildren<3->{n5}{n3}{n2};
		\draw<4>[-triangle 45,thick] (n2) -- (n5);
	\end{tikzpicture}
\end{frame}

\begin{frame}
	\frametitle{Top-down Subsumption Example}
	\centering
	\begin{tikzpicture}[node distance=1.2cm]
		\rootnode;
		\withchildren{root} {n5}{$a$}  {n7}{$\dual{a}$};
	\end{tikzpicture}
\end{frame}

\begin{frame}
	\frametitle{Bottom-up Subsumption Example}
	\centering
	\begin{tikzpicture}[node distance=1.2cm]
		\rootnode;
		\withchildren{root} {n5}{$\dual{a}$}  {n6}{$a$};
		\withchildren{n5} {n1}{$\dual{b}$} {n4}{$b$};
		\withchildren{n4} {n2}{$\dual{a},b$} {n3}{$a,b$};
		\only<1>{
		\node [color=red] (numb1) at (n1.west) {\tiny{1}};
		\node [color=red] (numb2) at (n2.west) {\tiny{2}};
		\node [color=red] (numb3) at (n3.west) {\tiny{3}};
		\node [color=red] (numb4) at (n4.west) {\tiny{4}};
		\node [color=red] (numb5) at (n5.west) {\tiny{5}};
		\node [color=red] (numb6) at (n6.west) {\tiny{6}};
		\node [color=red] (numb7) at (root.west) {\tiny{7}};
		}
		\marknode<2>{n3};
		\marknode<2>{n6};
	\end{tikzpicture}
\end{frame}


\begin{frame}
	\frametitle{Bottom-up Subsumption Example}
	\centering
	\begin{tikzpicture}[node distance=1.2cm]
		\rootnode;
		
		\proofnode [above left of= root]{n5} {$\dual{a}$};
		\withchildren{n5} {n1}{$\dual{b}$} {n4}{$b$};
		
		\withchildren{n4} {n2}{$\dual{a},b$} {n6}{$a$};
		
		\drawchildren{root}{n5}{n6};

	\end{tikzpicture}
\end{frame}

\begin{frame}
	\frametitle{Bottom-up Subsumption Example}
	\centering
	\begin{tikzpicture}[node distance=1.2cm]
		\rootnode;
		\withchildren{root} {n5}{$\dual{a}$}  {n6}{$a$};
		\withchildren{n5} {n1}{$\dual{b}$} {n4}{$b$};
		\withchildren{n4} {n2}{$\dual{a},b$} {n3}{$a,b$};
		\only<1>{
		\node [color=red] (numb1) at (n1.west) {\tiny{1}};
		\node [color=red] (numb2) at (n2.west) {\tiny{2}};
		\node [color=red] (numb3) at (n3.west) {\tiny{3}};
		\node [color=red] (numb4) at (n4.west) {\tiny{4}};
		\node [color=red] (numb5) at (n5.west) {\tiny{5}};
		\node [color=red] (numb6) at (n6.west) {\tiny{6}};
		\node [color=red] (numb7) at (root.west) {\tiny{7}};
		}
		\marknode<2>{n2};
		\marknode<2>{n5};
	\end{tikzpicture}
\end{frame}

\begin{frame}
	\frametitle{Bottom-up Subsumption Example}
	\centering
	\begin{tikzpicture}[node distance=1.2cm]
		\rootnode;
		\withchildren{root} {n5}{$\dual{a}$}  {n6}{$a$};
		\withchildren{n5} {n1}{$\dual{b}$} {n4}{$b$};
		\proofnode[above right of=n4]{n3} {$a,b$};
		\draw [proof edge] (n4) -- (n3);
		\draw [->,proof edge] (n5) -- (n4);
		\draw [->,proof edge](n4) [out = 150, in = 90] to (n5);
	\end{tikzpicture}
\end{frame}

\begin{frame}

\frametitle{Problems with Bottom-up Subsumption}

\begin{subpart}{Performance}
	\item Ancestor check is costly
\end{subpart}

%proof with cyclic dependency

\begin{subpart}{Copy node issue}
	\item Fixing has to be done top-down
	\item Nodes are replaced bottom-up
\end{subpart}

%illustrate refixing problem

\end{frame}

\begin{frame}

\frametitle{RecycleUnits}

\begin{subpart}{Omer Bar-Ilan et al., 2009}
	\item IBM Haifa Research Laboratory
\end{subpart}

\begin{subpart}{Special case of bottom-up subsumption}
	\item Check only for subsuming unit clauses
	\item Replace subsumption check by comparing units to pivots
	\item Worst case quadratic runtime in the amount of unit clauses
\end{subpart}

%problems??

\end{frame}

\begin{frame}

	\frametitle{Experiments}

	\begin{subpart}{Setting}
		\item 500 proofs, provided by VeriT SMT Solver
		\item 659,584 nodes in total
		\item Average 1320 nodes per proof
	\end{subpart}

	\begin{block}{Results}
		\vspace{1em}
		\centering
		\begin{tabular}{lcc}
			\hline
			Algorithm                & Length Compression & Speed   \\
			\hline
			Top-down Subsumption     &  3.7 \% & 2.3 nodes/ms \\
			Bottom-up Subsumption    &  1\%    & 0.4 nodes/ms \\
			RecycleUnits						 &  2\%    & 1.0 nodes/ms \\
			DAGify          				 &  0.6 \% & 7.3 nodes/ms \\  
			\hline
		\end{tabular}
	\end{block}

\end{frame}

\section{Recursive Split}
\recalltoc
\begin{frame}

\frametitle{Split Algorithm}

\begin{subpart}{Author}
	\item Scott Cotton, 2010
\end{subpart}
\begin{subpart}{Idea}
	\item Let $P$ be a proof with root clause $C$
	\item Choose a variable $v$, occuring in $P$, heuristically
	\item Extract proofs for $C \vee v$ and $C \vee \dual{v}$
	\begin{itemize}
		\item By deleting positive/negative branches of nodes with pivot $v$
		\item And fixing nodes like at subsumption algorithms
	\end{itemize}
	\item Combine proofs by resolving roots
\end{subpart}

\end{frame}

\begin{frame}
	\frametitle{Split Algorithm Example}
	\centering
	%\vspace{-1cm}
	Split variable $a$\\[1cm]
	%\begin[center]{block}
	\begin{tikzpicture}[node distance=1.2cm]
		\rootnode;
		\withchildren{root} {n7}{$b$}  {n9}{$\dual{b}$};
		\withchildren{n7}   {n1}{$\dual{a},b$} {n6}{$a$};
		\proofnode[above right of=n9] {n8} {$\dual{b},\dual{a}$};
		\drawchildren{n9} {n6} {n8};
		\withchildren{n6} {n4}{$\dual{c}$} {n5}{$a,c$};
		\withchildren{n4} {n2}{$\dual{a},\dual{c}$}{n3}{$a,\dual{c}$};
	\end{tikzpicture}
	%\end{block}
\end{frame}

\begin{frame}
\frametitle{Split Algorithm Example}
	\only<1>{\centering{Duplicate the proof}\\[1cm]}
  \begin{columns}
  \column{0.4\textwidth}
	\only<2>{Delete positive branch\\[1cm]}
	\only<3>{Fix positive branch\\[1cm]}
	\begin{tikzpicture}[node distance=1.2cm]
		\rootnode;
		\withchildren{root} {n7}{$b$}  {n9}{$\dual{b}$};
		\withchildren{n7}   {n1}{$\dual{a},b$} {n6}{$a$};
		\proofnode[above right of=n9] {n8} {$\dual{b},\dual{a}$};
		\drawchildren{n9} {n6} {n8};
		\withchildren{n6} {n4}{$\dual{c}$} {n5}{$a,c$};
		\withchildren{n4} {n2}{$\dual{a},\dual{c}$}{n3}{$a,\dual{c}$};
		\crossnode<2->{n6};
		\draw<3>[-triangle 45] (n1) -- (n7);
		\draw<3>[-triangle 45] (n8) -- (n9);
	\end{tikzpicture}
	\column{0.4\textwidth}
	\only<2>{Delete negative branch\\[1cm]}
	\only<3>{Fix negative branch\\[1cm]}
	\begin{tikzpicture}[node distance=1.2cm]
		\rootnode;
		\withchildren{root} {n7}{$b$}  {n9}{$\dual{b}$};
		\withchildren{n7}   {n1}{$\dual{a},b$} {n6}{$a$};
		\proofnode[above right of=n9] {n8} {$\dual{b},\dual{a}$};
		\drawchildren{n9} {n6} {n8};
		\withchildren{n6} {n4}{$\dual{c}$} {n5}{$a,c$};
		\withchildren{n4} {n2}{$\dual{a},\dual{c}$}{n3}{$a,\dual{c}$};
		\crossnode<2->{n2};
		\crossnode<2->{n1};
		\crossnode<2->{n8};
		\draw<3>[-triangle 45] (n3) -- (n4);
		\draw<3>[-triangle 45] (n6) -- (n7);
		\draw<3>[-triangle 45] (n6) -- (n9);
	\end{tikzpicture}

	\end{columns}
\end{frame}

\begin{frame}
\frametitle{Split Algorithm Example}
  \begin{columns}
  \column{0.4\textwidth}
	Fix positive branch\\[1cm]
	\begin{tikzpicture}[node distance=1.2cm]
		\rootnode;
		\withchildren{root} {n1}{$\dual{a},b$}  {n9}{$\dual{b},\dual{a}$};
	\end{tikzpicture}
	\column{0.4\textwidth}
	Fix negative branch\\[1cm]
	\begin{tikzpicture}[node distance=1.2cm]
		\rootnode;
		\proofnode[above of=root]{n6} {a};
		\withchildren{n6} {n4}{$a,\dual{c}$} {n5}{$a,c$};
		\draw [proof edge,bend left=45] (root) to (n6);
		\draw [proof edge,bend right=45] (root) to (n6);
	\end{tikzpicture}

	\end{columns}
\end{frame}

\begin{frame}
\frametitle{Split Algorithm Example}
  \begin{columns}
  \column{0.4\textwidth}
	Fix positive branch\\[1cm]
	\begin{tikzpicture}[node distance=1.2cm]
		\proofnode{n6} {$\dual{a}$};
		\withchildren{n6} {n1}{$\dual{a},b$}  {n9}{$\dual{b},\dual{a}$};
	\end{tikzpicture}
	\column{0.4\textwidth}
	Fix negative branch\\[1cm]
	\begin{tikzpicture}[node distance=1.2cm]
		\proofnode{n6} {a};
		\withchildren{n6} {n4}{$a,\dual{c}$} {n5}{$a,c$};
	\end{tikzpicture}

	\end{columns}
\end{frame}

\begin{frame}
\frametitle{Split Algorithm Example}
	\centering
	Combine branches by resolving\\[1cm]
	\begin{tikzpicture}[node distance=1.2cm]
		\rootnode;
		%\withchildren{root} {n3}{$\dual{a}$} {n5}{$a$};
		\proofnode[above left of =root,xshift=-1cm]{n3}{$\dual{a}$};
		\proofnode[above right of =root,xshift=1cm]{n5}{$a$};
		\drawchildren{root} {n3} {n5};
		\withchildren{n3} {n1}{$\dual{a},b$}  {n2}{$\dual{b},\dual{a}$};
		\withchildren{n5} {n4}{$a,\dual{c}$} {n6}{$a,c$};
	\end{tikzpicture}
\end{frame}

\begin{frame}

\frametitle{Iterative Split}

\begin{subpart}{Idea}
	\item Apply split to result of split
\end{subpart}

\begin{subpart}{Issues}
	\item Best variables don't end up lowest
	\item Same variables for positive and negative branches
\end{subpart}
\begin{block}{Example}
\end{block}
  \begin{columns}
  \column{0.5\textwidth}
	\centering
	Split once, best variable: $a$\\
	\begin{tikzpicture}
		\rootnode;
		\withchildren{root}{n1}{$a$} {n2}{$\dual{a}$};
		\addchildren{n1} {trans1}{} {trans2}{};
		\addchildren{n2} {trans3}{} {trans4}{};
		\draw [dashed] (n1) to (trans1);
		\draw [dashed] (n1) to (trans2);
		\draw [dashed] (n2) to (trans3);
		\draw [dashed] (n2) to (trans4);
	\end{tikzpicture}
	\column{0.5\textwidth}
	\centering
	Split again, best variable: $b$\\
	\begin{tikzpicture}
		\rootnode;
		\proofnode[above left of = root,xshift = -0.6cm] {n1}{$b$};
		\proofnode[above right of = root,xshift = 0.6cm] {n2}{$\dual{b}$};
		\drawchildren{root}{n1} {n2};
		\withchildren{n1} {n3}{$b,a$} {n4}{$b,\dual{a}$};
		\withchildren{n2} {n5}{$\dual{b},a$} {n6}{$\dual{b},\dual{a}$};
		\addchildren{n3} {trans1}{} {trans2}{};
		\addchildren{n5} {trans3}{} {trans4}{};
		\addchildren{n6} {notneeded}{} {trans5}{};
		\draw [dashed] (n3) to (trans1);
		\draw [dashed] (n3) to (trans2);
		\draw [dashed] (n4) to (trans2);
		\draw [dashed] (n4) to (trans3);
		\draw [dashed] (n5) to (trans3);
		\draw [dashed] (n5) to (trans4);
		\draw [dashed] (n6) to (trans4);
		\draw [dashed] (n6) to (trans5);
	\end{tikzpicture}
\end{columns}

\end{frame}

\begin{frame}
\frametitle{Recursive Split}
\begin{subpart}{Idea}
	\item Apply split to positive and negative branches before combining
	\item Use stopping criteria (depth or time)
\end{subpart}

\begin{block}{Example}
\end{block}
\vspace{-0.8cm}
  \begin{columns}
  \column{0.5\textwidth}
	\centering
	Split once, best variable: $a$\\
	\begin{tikzpicture}
		\rootnode;
		\withchildren{root}{n1}{$a$} {n2}{$\dual{a}$};
		\addchildren{n1} {trans1}{} {trans2}{};
		\addchildren{n2} {trans3}{} {trans4}{};
		\draw [dashed] (n1) to (trans1);
		\draw [dashed] (n1) to (trans2);
		\draw [dashed] (n2) to (trans3);
		\draw [dashed] (n2) to (trans4);
	\end{tikzpicture}
	\column{0.5\textwidth}
	\centering
	Split positive/negative branches, best variables: $b$/$c$; \\
	\begin{tikzpicture}
		\rootnode;
		\proofnode[above left of = root,xshift = -0.6cm] {n1}{$a$};
		\proofnode[above right of = root,xshift = 0.6cm] {n2}{$\dual{a}$};
		\drawchildren{root}{n1} {n2};
		\withchildren{n1} {n3}{$a,b$} {n4}{$a, \dual{b}$};
		\withchildren{n2} {n5}{$\dual{a},c$} {n6}{$\dual{a},\dual{c}$};
		\addchildren{n3} {trans1}{} {trans2}{};
		\addchildren{n5} {trans3}{} {trans4}{};
		\addchildren{n6} {notneeded}{} {trans5}{};
		\draw [dashed] (n3) to (trans1);
		\draw [dashed] (n3) to (trans2);
		\draw [dashed] (n4) to (trans2);
		\draw [dashed] (n4) to (trans3);
		\draw [dashed] (n5) to (trans3);
		\draw [dashed] (n5) to (trans4);
		\draw [dashed] (n6) to (trans4);
		\draw [dashed] (n6) to (trans5);
	\end{tikzpicture}
\end{columns}
\vspace{-0.3cm}
\begin{subpart}{Issue}
	\item Equal nodes may be computed in both branches
\end{subpart}

\end{frame}

\begin{frame}

\frametitle{Experiments}

	\begin{block}{Results}
		\vspace{1em}
		\centering
		\begin{tabular}{lcc}
			\hline
			Algorithm                & Length Compression & Speed   \\
			\hline
			Recursive Split (depth 3) &  2.3\% & 3.0 nodes/ms \\
			Recursive Split (depth 5) &  2.0\% & 2.0 nodes/ms \\
			Iterative Split (depth 3) &  2.9\% & 3.2 nodes/ms \\
			Iterative Split (depth 5) &  3.4\% & 2.3 nodes/ms \\
			\hline
		\end{tabular}
	\end{block}

\end{frame}

\section{Greedy pebbling}
\recalltoc
\begin{frame}

\frametitle{Space Compression}

\begin{subpart}{Space measure}
	\item Maximal amount of nodes that have to be kept in memory at once
\end{subpart}

\vspace{-2mm}
\begin{subpart}{Deletion information}
	\item Extra lines in proof output
	\item Example: $y$ is the last child of $x$
	\begin{itemize}
		\item Read and check node $x$\\
		$\ldots$
		\item Read and check node $y$
		\item Delete node $x$\\
		$\ldots$
	\end{itemize}
\end{subpart}

\vspace{-4mm}
\begin{subpart}{Interesting scenario}
	\item Proof checker has much less memory than proof producer
\end{subpart}

\end{frame}

\begin{frame}

\frametitle{Black Pebbling Game}

\textit{A pebble is a small stone}

%I think to pebble is not a real word, but it's shorter and easiert than "put a pebble on"
\begin{subpart}{Rules}
	\item If all premises of a node $p$ are pebbled, $p$ may be pebbled
	\begin{itemize}
		\item In this case, a pebble may be moved from a premise to $p$
	\end{itemize}
	\item Nodes can be unpebbled at any time
\end{subpart}

\begin{subpart}{Goal}
	\item Pebble some node $v$
\end{subpart}

\begin{subpart}{Pebbling problem}
	\item For a given DAG and a node $v$, can $v$ be pebbled using no more than $n$ pebbles in total?
	\item PSPACE-complete (John R. Gilbert et al., 1980)
\end{subpart}

\end{frame}

\begin{frame}
	\frametitle{Pebbling Game Example}
	\centering
	Maximum pebbles used:
		\alt<1>{0}{\alt<2>{1}{\alt<3,4>{2}{3}}} \\[1cm]
	\begin{tikzpicture}[node distance=1.2cm]
		\rootnodeBW;
		\withchildrenBW{root}{n5}{n7};
		\withchildrenBW{n5}{n3}{n4};
		\withchildrenBW{n3}{n1}{n2};
		\proofnodeBW[above right of=n4]{n8};
		\drawchildren{n4}{n2}{n8};
		\proofnodeBW[above right of=n7]{n6};
		\drawchildren{n7}{n4}{n6};
		\blacknode<2,3>{n1};
		\blacknode<3-5>{n2};
		\blacknode<4-6>{n3};
		\blacknode<5>{n8};
		\blacknode<6-8>{n4};
		\blacknode<7-9>{n5};
		\blacknode<8>{n6};
		\blacknode<9>{n7};
		\blacknode<10>{root};
	\end{tikzpicture}
\end{frame}

\begin{frame}
	\frametitle{Greedy Pebbling}
	
	\begin{subpart}{Topological Order + Deletion Information}
		\item Correspond to a strategy for the pebbling game
	\end{subpart}
	
	\begin{subpart}{Greedy heuristics}
		\item Find better order w.r.t. space measure
		\item Choose next node w.r.t.:
		\begin{itemize}
			\item Number of pebbles that can be removed
			\item Pebbled premises
			\item Children with pebbled premises
			\item Number of children
			\item Number of premises, which the node is the last children of
			\item $\ldots$
		\end{itemize}
	\end{subpart}
	
	\begin{block}{Top-down vs Bottom-up}
	\end{block}
	
\end{frame}

\begin{frame}
	\frametitle{Greedy Top-down Pebbling}
	\centering
	Maximum pebbles used:
		\alt<1>{0}{\alt<2>{1}{\alt<3,4,5>{2}{\alt<6,7,8>{3}{4}}}} \\[1cm]
	\begin{tikzpicture}[node distance=1.2cm]
		\proofnodeBWHidden{root};
		\proofnodeBWHidden[above left of = root,xshift=-1.2cm]{n7};
		\proofnodeBW[above right of = n7,xshift=1.2cm]{n6};
		\proofnodeBW[above left of = n7]{n3};
		\proofnodeBWHidden[above right of = root,xshift=1.2cm]{n12};
		\proofnodeBW[above right of = n12]{n10};
		\withchildrenBW{n3}{n1}{n2};
		\withchildrenBW{n6}{n4}{n5};
		\withchildrenBW{n10}{n8}{n9};
		\blacknode<2,3>{n1};
		\blacknode<3>{n2};
		\blacknode<4-10>{n3};
		\whitenode<3->{n7};
		\draw<3->[proof edge] (n7.north west) -- (n3.south east);
		\blacknode<5,6>{n8};
		\blacknode<6>{n9};
		\blacknode<7-11>{n10};
		\whitenode<6->{n12};
		\draw<6->[proof edge] (n10) -- (n12);
		\blacknode<8,9>{n4};
		\blacknode<9>{n5};
		\blacknode<10,11>{n6};
		\whitenode<10->{root};
		\draw<9->[proof edge] (n6.south west) -- (n7.north east);
		\draw<9->[proof edge] (n6.south east) -- (n12.north west);
		\draw<10->[proof edge] (n7.south east) -- (root.north west);
		\draw<10->[proof edge] (n12.south west) -- (root.north east);	
		\blacknode<11,12>{n7};
		\blacknode<12>{n12};
		\blacknode<13>{root};
	\end{tikzpicture}
\end{frame}

\begin{frame}
	\frametitle{Greedy Bottom-up Pebbling}
	\centering
	Maximum pebbles used:
		\alt<1,2,3>{0}{\alt<4>{1}{\alt<5,6,7,8>{2}{3}}} \\[1cm]
	\begin{tikzpicture}[node distance=1.2cm]
		\proofnodeBW{root};
		\proofnodeBW[above left of = root,xshift=-1.2cm]{n7};
		\proofnodeBW[above right of = root,xshift=1.2cm]{n12};
		\draw[proof edge] (n7.south east) -- (root.north west);
		\draw[proof edge] (n12.south west) -- (root.north east);	
		
		\proofnodeBWHidden[above right of = n7,xshift=1.2cm]{n6};
		\proofnodeBWHidden[above left of = n7]{n3};
		\proofnodeBWHidden[above right of = n12]{n10};
		
		\proofnodeBWHidden[above right of = n3]{n2};
		\proofnodeBWHidden[above left of = n3]{n1};
		
		\proofnodeBWHidden[above right of = n6]{n5};
		\proofnodeBWHidden[above left of = n6]{n4};
		
		\proofnodeBWHidden[above right of = n10]{n8};
		\proofnodeBWHidden[above left of = n10]{n9};
		
		\waitingnode<1->{root};
		\waitingnode<2-17>{n12};
		\whitenode<2->{n6};
		\whitenode<2->{n10};
		\draw<2->[proof edge] (n6.south east) -- (n12.north west);
		\draw<2->[proof edge] (n10) -- (n12);
		\waitingnode<3-6>{n6};
		\whitenode<3->{n4};
		\whitenode<3->{n5};
		\drawchildren<3->{n6}{n4}{n5};
		\blacknode<4,5>{n5};
		\blacknode<5>{n4};
		\blacknode<6-10>{n6};
		\waitingnode<7-9>{n10};
		\whitenode<7->{n8};
		\whitenode<7->{n9};
		\drawchildren<7->{n10}{n8}{n9};
		\blacknode<8,9>{n8};
		\blacknode<9>{n9};
		\blacknode<10>{n10};
		\blacknode<11-17>{n12};
		\waitingnode<12-17>{n7};
		\whitenode<12->{n3};
		\draw<12->[proof edge] (n7.north west) -- (n3.south east);
		\draw<12->[proof edge] (n6.south west) -- (n7.north east);
		
		\waitingnode<13-15>{n3};
		\drawchildren<13->{n3}{n1}{n2};
		\whitenode<13->{n1};
		\whitenode<13->{n2};
		
		\blacknode<14,15>{n1};
		\blacknode<15>{n2};
		
		\blacknode<16>{n3};
		
		\blacknode<17>{n7};
		
		\blacknode<18>{root};
	\end{tikzpicture}
\end{frame}


\begin{frame}

\frametitle{Experiments}

	\begin{block}{Results}
		\vspace{1em}
		\centering
		\begin{tabular}{lcc}
			\hline
			Algorithm                & Space Compression & Speed   \\
			\hline
			Top-down Pebbling*       &  -40.4776\% & 1.4 nodes/ms \\
			Bottom-up Pebbling**     &  6.8\%    & 12.2 nodes/ms \\
			\hline
		\end{tabular}
	\end{block}

\begin{subpart}{Used heuristics}
	\item *: removes pebbles $>$ pebbled premises $>$ children with pebbled premises $>$ index in the original proof
	\item **: last child of a node $>$ number of children $>$ index in the original proof
\end{subpart}
\end{frame}

\begin{frame}
  \frametitle{Conclusion \& Future Work}
	\begin{subpart}{Subsumption}
    \item Top-down-S interesting replacement for DAGification
		\item Bottom-up-S could possibly do much
		\item Boost performance
		\item Fix problems with Bottom-up-S
  \end{subpart}
  \vspace{-0.1cm}
	\begin{subpart}{Recursive Split}
    \item Promising Idea
		\item Still needs some fine tuning
  \end{subpart}
	\vspace{-0.1cm}
	\begin{subpart}{Greedy Pebbling}
		\item Top-down version is not clever enough
		\item Bottom-up version shows nice results
		\item Find better heuristics for Bottom-up
	\end{subpart}
\end{frame}

\begin{frame}
	\begin{center}
		Special Thanks to:\\[5mm]
		Google \&\\[5mm]
		European Master's Program in Computational Logic (EMCL)\\[5mm]
		for financial support
	\end{center}
\end{frame}

\begin{frame}
	\vspace{2cm}
	\begin{center}
		Thank you for your attention\\[5mm]
		Feel free to ask questions\\[25mm]
		http://github.com/Paradoxika/Skeptik
	\end{center}
\end{frame}

\end{document}