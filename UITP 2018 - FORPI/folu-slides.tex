%\section{Propositional Algorithm}

\begin{frame}{Lowering Units}
\begin{definition}[Unit]
A unit clause is a subproof with a conclusion clause (final clause) having exactly 1 literal
\end{definition}
\begin{theorem}[\cite{fontaine2011compression}]
A unit clause can always be lowered
\end{theorem}
\vspace{0.5cm}
Compression is achieved by delaying resolution with unit clause subproofs.\\
\vspace{0.5cm}


%brief high level description; complexity\\
%probably not pseudo-code
\end{frame}

\begin{frame}{LowerUnits}
Two Traversals of the proof
\begin{itemize}
\item $\uparrow$ Collect units with more than one resolvent
\item $\downarrow$ Delete units and reintroduce them at the bottom of the proof
\end{itemize}

\end{frame}

\begin{frame}{Propositional Example}
%quick, clear example of LU (animated), perhaps showing how one of the redundancies described before is fixed
\begin{columns}
\column{0.5\textwidth}
      \begin{center}
      \begin{tikzpicture}[node distance=1.2cm]
        \rootnode;
        \withchildren{root} {r9}{\dual{c}}  {r6}{c};
     \proofnode[above left of=r9] {r8} {a,\dual{c}};
     \proofnode[above of=r8] {r7} {a,\dual{b},\dual{c}};
     \proofnode[above right of=r6] {r5} {a,c};
      \draw[proof edge] (r9) -- (r1);
        \withchildren{r5}  {r4}{b} {r2}{a,\dual{b},c};
        \withchildren{r4}   {r1}{\dual{a}} {r3}{a,b};
        \drawchildren {r6} {r1} {r5};
      \draw[proof edge] (r9) -- (r1);
      \draw[proof edge] (r9) -- (r8);
      \draw[proof edge] (r8) -- (r7);
      \draw[proof edge] (r8) -- (r4);
      \end{tikzpicture}
\end{center}
\column{0.5\textwidth}
  \only<1-3>{
      \begin{center}
      \begin{tikzpicture}[node distance=1.2cm]
 \proofnode {root}{\alt<3>{a,\dual{b}}{$\bot$}};
%        \withchildren{root} {r9}{\dual{c}}  {r6}{c};
 \proofnode[above right of=root] {r6}{c};
 \proofnode[above left of=root] {r9}{\dual{c}};
     \proofnode[above right of=r6] {r5} {a,c};
     \proofnode[above right of=r5] {r2} {a,\dual{b},c};
     \proofnode[above left of=r5] {r4} { \alt<2->{{\color{red} b}}{b} };
     \proofnode[above left of=r4] {r1} { \alt<2->{{\color{red}\dual{a}}} {\dual{a}}};
     \proofnode[above right of=r4] {r3} {\alt<3>{{\color{red}a,b}}{a,b}};
     \proofnode[above left of=r9] {r8} {a,\dual{c}};
     \proofnode[above of=r8] {r7} {a,\dual{b},\dual{c}};

%\marknode<2>{r4};
%\marknode<2->{r1};
%\marknode<3>{r3};

      \draw<-1>[proof edge] (r9) -- (r1);
      \draw<-1>[proof edge] (r9) -- (r8);
      \draw<-1>[proof edge] (r8) -- (r7);
      \draw<-1>[proof edge] (r8) -- (r4);
      \draw<-2>[proof edge] (root) -- (r6);
      \draw<-2>[proof edge] (root) -- (r9);
      \draw<-1>[proof edge] (r6) -- (r1);
      \draw<-1>[proof edge] (r6) -- (r5);
      \draw<-1>[proof edge] (r5) -- (r4);
      \draw<-1>[proof edge] (r5) -- (r2);
      \draw<-1>[proof edge] (r4) -- (r1);
      \draw<-1>[proof edge] (r4) -- (r3);

      \draw<2>[deleted edge] (r9) -- (r1);
      \draw<-2>[proof edge] (r9) -- (r8);
      \draw<-2>[proof edge] (r8) -- (r7);
      \draw<2>[deleted edge] (r8) -- (r4);
      \draw<2>[deleted edge] (r6) -- (r1);
      \draw<-2>[proof edge] (r6) -- (r5);
      \draw<2>[deleted edge] (r5) -- (r4);
      \draw<-2>[proof edge] (r5) -- (r2);
      \draw<2>[deleted edge] (r4) -- (r1);
      \draw<2>[proof edge] (r4) -- (r3);

%      \draw<3>[proof edge] (r4) -- (r3);
      \draw<3>[proof edge] (root) .. controls (r9.north east) .. (r7);
      \draw<3>[proof edge] (root) .. controls (r5.north west) .. (r2);

\crossnode<3>{r9}
\crossnode<3>{r4}
\crossnode<3>{r6}
\crossnode<3>{r8}
\crossnode<3>{r5}
      \draw<3>[deleted edge] (r4) -- (r3);

%        \drawchildren {r6} {r1} {r5};

%      \draw<2->[deleted edge] (r1) -- (r6);
%      \draw<-3>[proof edge] (r9) -- (r6);
%      \draw<4>[deleted edge] (r9) -- (r6);
      \end{tikzpicture}
\end{center}
}
  \only<4-6>{
      \begin{center}
      \begin{tikzpicture}[node distance=1.2cm]
     \proofnode{n2} {\alt<6->{$\bot$}{}};
     \proofnode [above left of=n2]{n1} {\alt<5->{a}{}};
     \proofnode [above right of=n2]{n3} {\alt<6>{   \alt<6>{{\color{blue} \dual{a}} }{\dual{a}}} {}};
     \proofnode [above right of=n1]{n4} {\alt<5->{   \alt<5>{{\color{blue} a,b}}{a,b}           }{}};
 \proofnode [above left of=n1]{root}{\alt<3->{a,\dual{b}}{$\bot$}};
     \proofnode[above right of=root] {r2} {a,\dual{b},c};
     \proofnode[above left of=root] {r7} {a,\dual{b},\dual{c}};

      \draw<4->[proof edge] (root) --  (r7);
      \draw<4->[proof edge] (root) -- (r2);
%\marknode<5>{n4};
%\marknode<6>{n3};
      \draw<5->[proof edge] (n1) -- (root);
      \draw<5->[proof edge] (n1) -- (n4);
      \draw<6->[proof edge] (n2) -- (n1);
      \draw<6->[proof edge] (n2) -- (n3);
      \end{tikzpicture}
\end{center}
}
  \only<7>{
      \begin{center}
      \begin{tikzpicture}[node distance=1.2cm]
     \proofnode{n2} {\alt<6->{$\bot$}{}};
     \proofnode [above left of=n2]{n1} {\alt<5->{a}{}};
     \proofnode [above right of=n2]{n3} {\dual{a}};
     \proofnode [above right of=n1]{n4} {\alt<5->{   \alt<5>{{\color{blue} a,b}}{a,b}           }{}};
 \proofnode [above left of=n1]{root}{\alt<3->{a,\dual{b}}{$\bot$}};
     \proofnode[above right of=root] {r2} {a,\dual{b},c};
     \proofnode[above left of=root] {r7} {a,\dual{b},\dual{c}};

      \draw<4->[proof edge] (root) --  (r7);
      \draw<4->[proof edge] (root) -- (r2);
%\marknode<5>{n4};
%\marknode<6>{n3};
      \draw<5->[proof edge] (n1) -- (root);
      \draw<5->[proof edge] (n1) -- (n4);
      \draw<6->[proof edge] (n2) -- (n1);
      \draw<6->[proof edge] (n2) -- (n3);
      \end{tikzpicture}
\end{center}
}
\end{columns}
\end{frame}

%\section{First-Order Algorithm}

\begin{frame}{First-Order Change: Helpful Contractions}
\begin{columns}
\column{0.5\textwidth}
      \begin{center}
      \begin{tikzpicture}[node distance=2cm]
     \proofnode {root} {$\bot$};
     \proofnode[above right of=root] {n5} {$\eta_5$: $p(X)\e$};
 \proofnode[above left of=n5] {n3} {$\eta_3$: $\e q(Z)$};
 \proofnode[above right of=n5] {n4} {$\eta_4$: $p(X), q(Z)\e$};
 \proofnode[above right of=n3] {n1} {$\eta_1$: $p(W)\e q(Z)$};
 \proofnode[above left of=n3] {n2} { \alt<2->{{\color{red}$\eta_2$: $\e p(Y)$ }} {$\eta_2$: $\e p(Y)$}};
      \draw[proof edge] (root) .. controls (n3.south west) ..  (n2);
      \draw[proof edge] (root) -- (n5);
      \draw[proof edge] (n5) -- (n3);
      \draw[proof edge] (n5) -- (n4);
      \draw[proof edge] (n3) -- (n1);
      \draw[proof edge] (n3) -- (n2);
%\marknode<2->{n2};
      \end{tikzpicture}
\end{center}
\column{0.5\textwidth}
\begin{center}
\only<3->{
      \begin{tikzpicture}[node distance=2cm]
     \proofnode {n5} {$\eta_5': p(X),p(Y)\e$};
 \proofnode[above right of=n5] {n4} {$\eta_4'$: $p(X),q(Z)\e $};
 \proofnode[above left of=n5] {n1} {$\eta_1'$: $p(W)\e q(Z)$\hspace*{0.5cm}};
 \proofnode[below left of=n5] {n5c} {\alt<4->{$\lfloor\eta_5'\rfloor$: $p(U)\e$}{}};
 \proofnode[below right of=n5] {n2} {\alt<5->{$\eta_2'$: $\e p(Y)$}{}};
 \proofnode[below right of=n5c] {root} {\alt<5->{$\bot$}{}};
      \draw[proof edge] (n5) -- (n1);
      \draw[proof edge] (n5) -- (n4);
      \draw<4->[proof edge] (n5c) -- (n5);
      \draw<5->[proof edge] (root) -- (n2);
      \draw<5->[proof edge] (root) -- (n5c);

      \end{tikzpicture}
}
\end{center}
\end{columns}

\end{frame}

\begin{frame}{First-Order Challenge: Pre-Deletion Check}
%example 2 demonstrated; definition of pre-deletion unification property
%\begin{columns}
%\column{0.5\textwidth}
\begin{center}
      \begin{tikzpicture}[node distance=2cm,background rectangle/.style={fill=blue!10}, show background rectangle]
     \proofnode {root} {$\bot$};
 \proofnode[above right of=root] {n2} {$\eta_2$: $\e q(Y)$};
 \proofnode[above left of=root] {n1} {$\eta_1$: $q(Y) \e$};
 \proofnode[above left of=n1] {n5} {$\eta_5$: $q(Y) \e p(a)$};
 \proofnode[above right of=n1] {n4} {\alt<2->{{\color{red}$\eta_4$: $p(X) \e $ }}{$\eta_4$: $p(X) \e $}};
 \proofnode[above right of=n2] {n3} {$\eta_3$: $\e p(b),q(Y) $};
      \draw[proof edge] (root) -- (n1);
      \draw[proof edge] (root)  -- (n2);
      \draw[proof edge] (n1) -- (n5);
      \draw[proof edge] (n1) -- (n4);
      \draw[proof edge] (n2) -- (n4);
      \draw[proof edge] (n2) -- (n3);
%\marknode<2->{n4};
      \end{tikzpicture}\\
%\vspace{3cm}
%\end{center}
%\column{0.5\textwidth}
%\begin{center}
%\vspace*{2cm}
\only<3-4>{
      \begin{tikzpicture}[node distance=2.25cm]
     \proofnode {root} {$\eta$: $\e p(a),p(b)$};
 \proofnode[above left of=root] {n5} {$\eta_5'$: $q(Y) \e p(a)$};
 \proofnode[above right of=root] {n3} {$\eta_3'$: $\e p(b),q(Y) $};
 \proofnode[below=0.5 of root] {n} {\alt<4->{$\lfloor\eta \rfloor$}{}};
      \draw[proof edge] (root) -- (n5);
      \draw[proof edge] (root) -- (n3);
\draw<4->[proof edge] (n) -- (root);
\crossnode<4->{n}
      \end{tikzpicture}
}
\only<5->{
\begin{definition}[Pre-Deletion Property]
$\eta$ unit, $l\in \eta$, such that $l$ is resolved with literals $l_1,\ldots,l_n$ in a proof $\psi$. $\eta$ satisfies the \emph{pre-deletion unifiability} property in $\psi$ if $l_1,\ldots,l_n$ and $\overline{l}$ are unifiable.
\end{definition}
}
\end{center}
%\end{columns}

\end{frame}

\begin{frame}{First-Order Challenge: Post-Deletion Check}
%example 3 demonstrated; definition of post-deletion unification property
\begin{center}
      \begin{tikzpicture}[node distance=2.5cm,background rectangle/.style={fill=blue!10}, show background rectangle]

     \proofnode {root} {$\bot$};
% \proofnode[above right=0.5cm and 0.05cm of root] {n5} {$\eta_5$: $p(U,q(W,b)) \e$};
% \proofnode[above left=0.5cm and 0.05cm of n5] {n3} {$\eta_3$: $r(V),p(U,q(V,b)) \e$};
% \proofnode[above right=0.5cm and 1pt of n5] {n4} {$\eta_4$: $\e r(W)$};
% \proofnode[above right=0.5cm and 0.05cm of n3] {n1} {$\eta_1$: $r(Y),p(X,q(Y, b)),p(X,Y)\e $};
% \proofnode[above left=0.5cm and 0.05cm of n3] {n2} {$\eta_2$: $\e p(U,V)$};
 \proofnode[above left=0.25cm of root] {n5} {$\eta_5$: $p(U,q(W,b)) \e$};
 \proofnode[above right=0.25cm and -2cm of n5] {n3} {$\eta_3$: $r(V),p(U,q(V,b)) \e$};
 \proofnode[above left=0.25cm and -1cm of n5] {n4} {$\eta_4$: $\e r(W)$};
 \proofnode[above left=0.25cm and -2cm of n3] {n1} {$\eta_1$: $r(Y),p(X,q(Y, b)),p(X,Y)\e $};
 \proofnode[above right=0.25cm and -1cm of n3] {n2} {\alt<2-> {{\color{red} $\eta_2$: $\e p(U,V)$}} {$\eta_2$: $\e p(U,V)$}};
      \draw[proof edge] (root) .. controls (n3.south east) .. (n2);
      \draw[proof edge] (root) -- (n5);
      \draw[proof edge] (n5) -- (n4);
      \draw[proof edge] (n5) -- (n3);
      \draw[proof edge] (n3) -- (n2);
      \draw[proof edge] (n3) -- (n1);
%\marknode<2->{n1};
      \end{tikzpicture}\\
\only<3-4>{
\vspace{0.5cm}
      \begin{tikzpicture}[node distance=1.5cm]
 \proofnode {n5} {$\eta_5'$: $p(X,q(W,b)), p(X,W) \e$};
 \proofnode[above right of=n5] {n1} {$\eta_1'$: $r(Y),p(X,q(Y, b)),p(X,Y)\e $};
 \proofnode[left=1cm of n1] {n4} {$\eta_4'$: $\e r(W)$};
 \proofnode[below=0.5cm of n5] {n} {\alt<4->{$\lfloor \eta_5'\rfloor$}{}};

      \draw[proof edge] (n5) -- (n1);
      \draw[proof edge] (n5) -- (n4);
      \draw<4->[proof edge] (n) -- (n5);
\crossnode<4->{n}

      \end{tikzpicture}
}
\only<5->{
\begin{definition}[Post-Deletion Property]
$\eta$ unit, $l\in \eta$, such that $l$ is resolved with literals $l_1,\ldots,l_n$ in a proof $\psi$. $\eta$ satisfies the \emph{post-deletion unifiability} property in $\psi$ if $l_1^{\dagger\downarrow},\ldots,l_n^{\dagger\downarrow}$ and $\overline{l^\dagger}$ are unifiable, where $l^\dagger$ is the literal in $\psi'=\psi\setminus\{\eta\}$ corresponding to $l$ in $\psi$, and $l^{\dagger\downarrow}$ is the descendant of $l^\dagger$ in the roof of $\psi'$.
\end{definition}
}
\end{center}
\end{frame}

\begin{frame}{First-Order Lower Units Challenges}
%briefly mention all ideas, e.g. quadratic time naive approach to deal with both properties
%Doesn't appear linear: can't handle pre-deletion and post-deletion in a constant number of traversals

\begin{itemize}
\item Deletion changes literals
\item Unit collection depends on whether contraction is possible after propagation down the proof
\end{itemize}
\vspace{0.5cm}
Deletion of units require knowledge of proof after deletion, and deletion depends on what will be lowered.
\vspace{0.5cm}
\begin{itemize}
\item $O(n^2)$ solution to have full knowledge
\item Difficult bookkeeping required for implementation
\end{itemize}

\end{frame}

\begin{frame}{Greedy First-Order Lower Units - A Quicker Alternative}
%introduce simpler idea, make compromises explicit and list benefits\\
%high level description\\
%(probably not pseudo-code, but list \# of traversals, complexity, etc)
\begin{itemize}
 \item Ignore post-deletion satisfaction 
 \item Focus on pre-deletion satisfaction 
 \item Greedy contraction
\end{itemize}
\vspace{0.5cm}
\pause
Faster run-time (linear; one traversal)\\
Easier to implement
\vspace{0.5cm}
\pause
\begin{itemize}
\item Doesn't always compress (returns original proof sometimes)
\end{itemize}
\end{frame}

\begin{frame}{First-Order Example}
%small, animated example
\begin{center}
\only<1>{
      \begin{tikzpicture}[node distance=1.5cm]
     \proofnode {root} {$\bot$};
 \proofnode[above left of=root] {n10} {$\eta_{10}$: $p(a) \e$};
 \proofnode[above left of=n10] {n9} {$\eta_9$: $t(Z) \e$};
 \proofnode[above right of=n10] {n8} {\hspace{1cm}$\eta_8$: $p(a) \e t(Z)$};
 \proofnode[above left of=n8] {n7} {$\eta_7$: $q(X), p(a) \e t(Z)$\hspace*{1.0cm}};
 \proofnode[above right of=n8] {n6} {\hspace{1.5cm}$\eta_6$:  $\e q(X), t(Z) $};
 \proofnode[above left of=n6] {n5} {$\eta_5$:  $p(V) \e q(X), t(Z)$};
 \proofnode[above left of=n5] {n4} {$\eta_4$:  $r(X), p(V) \e q(Y), t(Z)$\hspace*{2.5cm}};
 \proofnode[above right of=n5] {n3} {\hspace{3cm}$\eta_3$:  $\e q(X), r(Y), t(Z) $};
 \proofnode[above left of=n3] {n2} {$\eta_2$:  $p(W) \e q(X), r(Y), t(Z)$\hspace*{2.0cm}};
 \proofnode[above right=0.5cm and 0cm of n3] {n1} {$\eta_1$:  $\e p(a) $};
      \draw[proof edge] (n8) -- (n7);
      \draw[proof edge] (n8) -- (n6);

      \draw[proof edge] (root) -- (n10);
      \draw[proof edge] (n10) -- (n9);
      \draw[proof edge] (n10) -- (n8);
      \draw[proof edge] (n6) -- (n5);
      \draw[proof edge] (n5) -- (n4);
      \draw[proof edge] (n5) -- (n3);
      \draw[proof edge] (n3) -- (n2);
      \draw[proof edge] (n3) -- (n1);
      \draw[proof edge] (n6) .. controls (n3.south east) .. (n1);
      \draw[proof edge] (root) .. controls (n6.south east) .. (n1);
%\marknode<2->{n1};
      \end{tikzpicture}

}
\only<2>{
      \begin{tikzpicture}[node distance=1.5cm]
     \proofnode {root} {$\bot$};
 \proofnode[above left of=root] {n10} {$\eta_{10}$: ${\color{blue}p(a)} \e$};
 \proofnode[above left of=n10] {n9} {$\eta_9$: $t(Z) \e$};
 \proofnode[above right of=n10] {n8} {\hspace{1cm}$\eta_8$: $p(a) \e t(Z)$};
 \proofnode[above left of=n8] {n7} {$\eta_7$: $q(X), p(a) \e t(Z)$\hspace*{1.0cm}};
 \proofnode[above right of=n8] {n6} {\hspace{1.5cm}$\eta_6$:  $\e q(X), t(Z) $};
 \proofnode[above left of=n6] {n5} {$\eta_5$:  ${\color{blue}p(V)} \e q(X), t(Z)$};
 \proofnode[above left of=n5] {n4} {$\eta_4$:  $r(X), p(V) \e q(Y), t(Z)$\hspace*{2.5cm}};
 \proofnode[above right of=n5] {n3} {\hspace{3cm}$\eta_3$:  $\e q(X), r(Y), t(Z) $};
 \proofnode[above left of=n3] {n2} {$\eta_2$:  ${\color{blue}p(W)} \e q(X), r(Y), t(Z)$\hspace*{2.0cm}};
 \proofnode[above right=0.5cm and 0cm of n3] {n1} {{\color{red} $\eta_1$:  $\e p(a) $ }};
      \draw[proof edge] (n8) -- (n7);
      \draw[proof edge] (n8) -- (n6);

      \draw[proof edge] (root) -- (n10);
      \draw[proof edge] (n10) -- (n9);
      \draw[proof edge] (n10) -- (n8);
      \draw[proof edge] (n6) -- (n5);
      \draw[proof edge] (n5) -- (n4);
      \draw[proof edge] (n5) -- (n3);
      \draw[proof edge] (n3) -- (n2);
      \draw[proof edge] (n3) -- (n1);
      \draw[proof edge] (n6) .. controls (n3.south east) .. (n1);
      \draw[proof edge] (root) .. controls (n6.south east) .. (n1);
%\marknode<2->{n1};
      \end{tikzpicture}
{\color{blue} $\sigma=\{W\rightarrow a, V\rightarrow a\}$}
}
\only<3>{

       \begin{tikzpicture}[node distance=1.5cm]
     \proofnode {root} {$\bot$};
 \proofnode[above left of=root] {n10} {$\eta_{10}$: $p(a) \e$};
 \proofnode[above left of=n10] {n9} {$\eta_9$: $t(Z) \e$};
 \proofnode[above right of=n10] {n8} {\hspace{1cm}$\eta_8$: $p(a) \e t(Z)$};
 \proofnode[above left of=n8] {n7} {$\eta_7$: $q(X), p(a) \e t(Z)$\hspace*{1.0cm}};
 \proofnode[above right of=n8] {n6} {\hspace{1.5cm}$\eta_6$:  $\e q(X), t(Z) $};
 \proofnode[above left of=n6] {n5} {$\eta_5$:  $p(V) \e q(X), t(Z)$};
 \proofnode[above left of=n5] {n4} {$\eta_4$:  $r(X), p(V) \e q(Y), t(Z)$\hspace*{2.5cm}};
 \proofnode[above right of=n5] {n3} {\hspace{3cm}$\eta_3$:  $\e q(X), r(Y), t(Z) $};
 \proofnode[above left of=n3] {n2} {$\eta_2$:  $p(W) \e q(X), r(Y), t(Z)$\hspace*{2.0cm}};
 \proofnode[above right=0.5cm and 0cm of n3] {n1} {{\color{red} $\eta_1$:  $\e p(a) $ }};
      \draw[proof edge] (n8) -- (n7);
      \draw[proof edge] (n8) -- (n6);

      \draw[proof edge] (root) -- (n10);
      \draw[proof edge] (n10) -- (n9);
      \draw[proof edge] (n10) -- (n8);
      \draw[proof edge] (n6) -- (n5);
      \draw[proof edge] (n5) -- (n4);
      \draw[proof edge] (n5) -- (n3);
      \draw[proof edge, color=blue] (n2) -- (n3) node [midway,left=0.5cm, fill=white] {$\{W\rightarrow a\}$};
      \draw[proof edge, color=red] (n3) -- (n1);
      \draw[proof edge] (n6) .. controls (n3.south east) .. (n1);
      \draw[proof edge] (root) .. controls (n6.south east) .. (n1);

      \end{tikzpicture}

}

\only<4>{

       \begin{tikzpicture}[node distance=1.5cm]
     \proofnode {root} {$\bot$};
 \proofnode[above left of=root] {n10} {$\eta_{10}$: $p(a) \e$};
 \proofnode[above left of=n10] {n9} {$\eta_9$: $t(Z) \e$};
 \proofnode[above right of=n10] {n8} {\hspace{1cm}$\eta_8$: $p(a) \e t(Z)$};
 \proofnode[above left of=n8] {n7} {$\eta_7$: $q(X), p(a) \e t(Z)$\hspace*{1.0cm}};
 \proofnode[above right of=n8] {n6} {\hspace{1.5cm}$\eta_6$:  $\e q(X), t(Z) $};
 \proofnode[above left of=n6] {n5} {{\color{blue}$\eta_5'$:  $p(V),p(a) \e q(X), t(Z)$}};
 \proofnode[above left of=n5] {n4} {$\eta_4$:  $r(X), p(V) \e q(Y), t(Z)$\hspace*{2.5cm}};
 \proofnode[above right of=n5] {n3} {\hspace{3cm}$\eta_3'$:  $p(a) \e q(X), r(Y), t(Z)$};
 \proofnode[above left of=n3] {n2} {\hspace*{2.0cm}};
 \proofnode[above right=0.5cm and 0cm of n3] {n1} {{\color{red} $\eta_1$:  $\e p(a) $ }};
      \draw[proof edge] (n8) -- (n7);
      \draw[proof edge] (n8) -- (n6);

      \draw[proof edge] (root) -- (n10);
      \draw[proof edge] (n10) -- (n9);
      \draw[proof edge] (n10) -- (n8);
      \draw[proof edge] (n6) -- (n5);
      \draw[proof edge] (n5) -- (n4);
      \draw[proof edge] (n5) -- (n3);
      \draw[proof edge] (n6) .. controls (n3.south east) .. (n1);
      \draw[proof edge] (root) .. controls (n6.south east) .. (n1);
%\marknode<4>{n5};
      \end{tikzpicture}

}

\only<5>{

       \begin{tikzpicture}[node distance=1.5cm]
     \proofnode {root} {$\bot$};
 \proofnode[above left of=root] {n10} {$\eta_{10}$: $p(a) \e$};
 \proofnode[above left of=n10] {n9} {$\eta_9$: $t(Z) \e$};
 \proofnode[above right of=n10] {n8} {\hspace{1cm}$\eta_8$: $p(a) \e t(Z)$};
 \proofnode[above left of=n8] {n7} {$\eta_7$: $q(X), p(a) \e t(Z)$\hspace*{1.0cm}};
 \proofnode[above right of=n8] {n6} {\hspace{1.5cm}$\eta_6$:  $\e q(X), t(Z) $};
 \proofnode[above left of=n6] {n5} {$\eta_5'$:  $p(V),p(a) \e q(X), t(Z)$};
 \proofnode[above left of=n5] {n4} {$\eta_4$:  $r(X), p(V) \e q(Y), t(Z)$\hspace*{2.5cm}};
 \proofnode[above right of=n5] {n3} {\hspace{3cm}$\eta_3$:  $p(a) \e q(X), r(Y), t(Z)$};
 \proofnode[above left of=n3] {n2} {\hspace*{2.0cm}};
 \proofnode[above right=0.5cm and 0cm of n3] {n1} {{\color{red} $\eta_1$:  $\e p(a) $ }};
      \draw[proof edge] (n8) -- (n7);
      \draw[proof edge] (n8) -- (n6);

      \draw[proof edge] (root) -- (n10);
      \draw[proof edge] (n10) -- (n9);
      \draw[proof edge] (n10) -- (n8);
      \draw[proof edge] (n5) -- (n4);
      \draw[proof edge] (n5) -- (n3);
      \draw[proof edge, color=blue] (n5) -- (n6) node [midway,left=0.5cm, fill=white] {$\{V\rightarrow a\}$};
      \draw[proof edge, color=red] (n6) .. controls (n3.south east) .. (n1);
      \draw[proof edge] (root) .. controls (n6.south east) .. (n1);

      \end{tikzpicture}

}

\only<6>{

       \begin{tikzpicture}[node distance=1.5cm]
     \proofnode {root} {$\bot$};
 \proofnode[above left of=root] {n10} {$\eta_{10}$: $p(a) \e$};
 \proofnode[above left of=n10] {n9} {$\eta_9$: $t(Z) \e$};
 \proofnode[above right of=n10] {n8} {\hspace{1cm}$\eta_8$: $p(a) \e t(Z)$};
 \proofnode[above left of=n8] {n7} {$\eta_7$: $q(X), p(a) \e t(Z)$\hspace*{1.0cm}};
 \proofnode[above=0.25cm of n6] {n6p} {{\color{blue}\hspace{3cm}$\eta_6'$:  $p(a), p(a) \e q(X), t(Z) $}};
 \proofnode[above right of=n8] {n6} {{\color{blue}\hspace{3cm}$\lfloor\eta_6'\rfloor$:   $p(a) \e q(X), t(Z)$}};
 \proofnode[above left of=n6] {n5} {};
 \proofnode[above left of=n5] {n4} {$\eta_4$:  $r(X), p(V) \e q(Y), t(Z)$\hspace*{2.5cm}};
 \proofnode[above right of=n5] {n3} {\hspace{3cm}$\eta_3$:  $p(a) \e q(X), r(Y), t(Z)$};
 \proofnode[above left of=n3] {n2} {\hspace*{2.0cm}};
 \proofnode[above right=0.5cm and 0cm of n3] {n1} {{\color{red} $\eta_1$:  $\e p(a) $ }};
      \draw[proof edge] (n8) -- (n7);
      \draw[proof edge] (n8) -- (n6);

      \draw[proof edge] (root) -- (n10);
      \draw[proof edge] (n10) -- (n9);
      \draw[proof edge] (n10) -- (n8);
      \draw[proof edge] (n6p) -- (n4);
      \draw[proof edge] (n6) -- (n6p);
      \draw[proof edge] (n6p) -- (n3);
      \draw[proof edge] (root) .. controls (n6.south east) .. (n1);

%\marknode<6>{n6};
      \end{tikzpicture}

}

\only<7>{

       \begin{tikzpicture}[node distance=1.5cm]
     \proofnode {root} {$\bot$};
 \proofnode[above left of=root] {n10} {$\eta_{10}$: $p(a) \e$};
 \proofnode[above left of=n10] {n9} {$\eta_9$: $t(Z) \e$};
 \proofnode[above right of=n10] {n8} {\hspace{1cm}$\eta_8$: $p(a) \e t(Z)$};
 \proofnode[above left of=n8] {n7} {$\eta_7$: $q(X), p(a) \e t(Z)$\hspace*{1.0cm}};
 \proofnode[above=0.25cm of n6] {n6p} {\hspace{3cm}$\eta_6'$:  $p(a), p(a) \e q(X), t(Z) $};
 \proofnode[above right of=n8] {n6} {\hspace{3cm}$\lfloor\eta_6'\rfloor$:   $p(a) \e q(X), t(Z)$};
 \proofnode[above left of=n6] {n5} {};
 \proofnode[above left of=n5] {n4} {$\eta_4$:  $r(X), p(V) \e q(Y), t(Z)$\hspace*{2.5cm}};
 \proofnode[above right of=n5] {n3} {\hspace{3cm}$\eta_3$:  $p(a) \e q(X), r(Y), t(Z)$};
 \proofnode[above left of=n3] {n2} {\hspace*{2.0cm}};
 \proofnode[above right=0.5cm and 0cm of n3] {n1} {{\color{red} $\eta_1$:  $\e p(a) $ }};
      \draw[proof edge] (n8) -- (n7);
      \draw[proof edge] (n8) -- (n6);

      \draw[proof edge, color=blue] (n10) -- (root) node [midway,left=0.5cm, fill=white] {$\emptyset$};
      \draw[proof edge] (n10) -- (n9);
      \draw[proof edge] (n10) -- (n8);
      \draw[proof edge] (n6p) -- (n4);
      \draw[proof edge] (n6) -- (n6p);
      \draw[proof edge] (n6p) -- (n3);
      \draw[proof edge, color=red] (root) .. controls (n6.south east) .. (n1);

%\marknode<7>{n8};
      \end{tikzpicture}

}

\only<8>{

       \begin{tikzpicture}[node distance=1.5cm]
     \proofnode {root} {$\eta_{10}$: $p(a) \e$};
 \proofnode[above left of=root] {n10} {};
 \proofnode[above left of=n10] {n9} {$\eta_9$: $t(Z) \e$};
 \proofnode[above right of=n10] {n8} {\hspace{1cm}$\eta_8$: $p(a) \e t(Z)$};
 \proofnode[above left of=n8] {n7} {$\eta_7$: $q(X), p(a) \e t(Z)$\hspace*{1.0cm}};
 \proofnode[above=0.25cm of n6] {n6p} {\hspace{3cm}$\eta_6'$:  $p(a), p(a) \e q(X), t(Z) $};
 \proofnode[above right of=n8] {n6} {\hspace{3cm}$\lfloor\eta_6'\rfloor$:   $p(a) \e q(X), t(Z)$};
 \proofnode[above left of=n6] {n5} {};
 \proofnode[above left of=n5] {n4} {$\eta_4$:  $r(X), p(V) \e q(Y), t(Z)$\hspace*{2.5cm}};
 \proofnode[above right of=n5] {n3} {\hspace{3cm}$\eta_3$:  $p(a) \e q(X), r(Y), t(Z)$};
 \proofnode[above left of=n3] {n2} {\hspace*{2.0cm}};
 \proofnode[above right=0.5cm and 0cm of n3] {n1} {{\color{red} $\eta_1$:  $\e p(a) $ }};
      \draw[proof edge] (n8) -- (n7);
      \draw[proof edge] (n8) -- (n6);


      \draw[proof edge] (root) -- (n9);
      \draw[proof edge] (root) -- (n8);
      \draw[proof edge] (n6p) -- (n4);
      \draw[proof edge] (n6) -- (n6p);
      \draw[proof edge] (n6p) -- (n3);


%\marknode<7>{n8};
      \end{tikzpicture}

}

\only<9>{

       \begin{tikzpicture}[node distance=1.5cm]
     \proofnode {root} {$\eta_{10}$: $p(a) \e$};
 \proofnode[above left of=root] {n10} {};
 \proofnode[above left of=n10] {n9} {$\eta_9$: $t(Z) \e$};
 \proofnode[above right of=n10] {n8} {\hspace{1cm}$\eta_8$: $p(a) \e t(Z)$};
 \proofnode[above left of=n8] {n7} {$\eta_7$: $q(X), p(a) \e t(Z)$\hspace*{1.0cm}};
 \proofnode[above=0.25cm of n6] {n6p} {\hspace{3cm}$\eta_6'$:  $p(a), p(a) \e q(X), t(Z) $};
 \proofnode[above right of=n8] {n6} {\hspace{3cm}$\lfloor\eta_6'\rfloor$:   $p(a) \e q(X), t(Z)$};
 \proofnode[above left of=n6] {n5} {};
 \proofnode[above left of=n5] {n4} {$\eta_4$:  $r(X), p(V) \e q(Y), t(Z)$\hspace*{2.5cm}};
 \proofnode[above right of=n5] {n3} {\hspace{3cm}$\eta_3$:  $p(a) \e q(X), r(Y), t(Z)$};
 \proofnode[above left of=n3] {n2} {\hspace*{2.0cm}};
 \proofnode[above right=0.5cm and 0cm of n3] {n1} {$\eta_1$:  $\e p(a) $};
     \proofnode[below right of=root] {newroot} {$\bot$};
      \draw[proof edge] (n8) -- (n7);
      \draw[proof edge] (n8) -- (n6);


      \draw[proof edge] (root) -- (n9);
      \draw[proof edge] (root) -- (n8);
      \draw[proof edge] (n6p) -- (n4);
      \draw[proof edge] (n6) -- (n6p);
      \draw[proof edge] (n6p) -- (n3);;
      \draw[proof edge] (newroot) .. controls (n6.south east) .. (n1);
      \draw[proof edge] (newroot) -- (root);

%\marknode<7>{n8};
      \end{tikzpicture}

}

\end{center}

\end{frame}

