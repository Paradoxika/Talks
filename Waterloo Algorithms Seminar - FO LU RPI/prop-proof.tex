

\begin{frame}{(Propositional) Proofs}
%introduction to proofs as we're going to see them, ideally formally and with a small example

\begin{definition}[Proof]
A directed acyclic graph $\langle V,E,\Gamma \rangle$, where
\begin{itemize}
\item $V$ is a set of nodes
\item $E$ is a set of edges labeled by literals
\item $\Gamma$ (the proof clause) is inductively constructible using \emph{axiom} and \emph{resolution} nodes
\end{itemize}
\end{definition}

\begin{definition}[Axiom]
A proof with a single node (so $E=\emptyset$)
\end{definition}
\end{frame}




\begin{frame}{(Propositional) Resolution}
%a quick introduction to resolution, with a small propositional example
\begin{definition}[Resolution]
Given two proofs $\psi_L$ and $\psi_R$ with conclusions $\Gamma_L$ and $\Gamma_R$ with some literal $l$ such that $\overline{l}\in \Gamma_L$ and $l\in \Gamma_R$, the resolution proof $\psi$ of $\psi_L$ and $\psi_R$ on $l$, denoted $\psi=\psi_L \psi_R$ is such that:
\begin{itemize}
\item $\psi$'s nodes are the union of the nodes of $\psi_L$ and $\psi_R$, and a new root node
\item there is an edge from $\rho(\psi)$ to $\rho(\psi_L)$ labeled with $\overline{l}$
\item there is an edge from $\rho(\psi)$ to $\rho(\psi_R)$ labeled with $l$
\item $\psi$'s conclusion is $(\Gamma_L\setminus\{\overline{l}\})\cup(\Gamma_R\setminus\{l\})$
\end{itemize}
\end{definition}
\end{frame}
